\documentclass[12pt,a4paper]{scrartcl}

% Grundlegende Einstellungen
\usepackage[ngerman]{babel}			% deutsche Namen/Umlaute
\usepackage[utf8]{inputenc}			% Zeichensatzkodierung
\usepackage{graphicx}				% Einbinden von Bildern
\usepackage{color}					% Farben wenn es sein muß
\usepackage[hidelinks]{hyperref}    % Klickbare Verweise und \autoref{label}
\providecommand*{\listingautorefname}{Listing}
\usepackage{bookmark}
\newcommand*{\email}[1]{            % Klickbare eMail-Adressen
    \normalsize\href{mailto:#1}{#1}\par
}
\usepackage{booktabs}				% "Schöne" Tabellen
\usepackage{amsmath}				% Mathematischer Formelsatz AMS
\usepackage{amsfonts}
\usepackage{csquotes}               % Empfohlener Import für babel/polyglossia
\usepackage{blindtext}              % Einfügen von Blindtext möglich

\usepackage[
    printonlyused                   % nur verwendete Abkürzungen ins Verzeichnis
]{acronym}                          % Abkürzungen
\usepackage[section]{placeins}      % Bilder bleiben im Abschnitt
% \usepackage{scrpage2}
% \usepackage{scrlayer-scrpage}       % Neuere Version von scrpage2
\usepackage[automark,headsepline,singlespacing=true]{scrlayer-scrpage}
\pagestyle{scrheadings}

% Neuer Befehl
% \clearmainofpairofpagestyles

% Aufzählungen anpassen wie hier beschrieben: https://www.latex-tutorial.com/tutorials/lists/
\usepackage{enumitem}

% Abkürzungen
\usepackage[acronym]{glossaries}

% Silbentrennung wie hier beschrieben: https://tex.stackexchange.com/a/119514
\usepackage[T1]{fontenc}

% Quellcode für Python
\usepackage{minted}
\newmintedfile[shellcode]{shell}{
fontfamily=tt,
linenos=true,
numberblanklines=true,
numbersep=5pt,
gobble=0,
frame=leftline,
framerule=0.4pt,
framesep=2mm,
funcnamehighlighting=true,
tabsize=4,
obeytabs=false,
mathescape=false
samepage=false, %with this setting you can force the list to appear on the same page
showspaces=false,
showtabs =false,
texcl=false,
}
\newmintedfile[pythoncode]{python}{
fontfamily=tt,
linenos=true,
numberblanklines=true,
numbersep=5pt,
gobble=0,
frame=leftline,
framerule=0.4pt,
framesep=2mm,
funcnamehighlighting=true,
tabsize=4,
obeytabs=false,
mathescape=false
samepage=false, %with this setting you can force the list to appear on the same page
showspaces=false,
showtabs =false,
texcl=false,
}
\newmintedfile[yamlcode]{yaml}{
fontfamily=tt,
linenos=true,
numberblanklines=true,
numbersep=5pt,
gobble=0,
frame=leftline,
framerule=0.4pt,
framesep=2mm,
funcnamehighlighting=true,
tabsize=4,
obeytabs=false,
mathescape=false
samepage=false, %with this setting you can force the list to appear on the same page
showspaces=false,
showtabs =false,
texcl=false,
breaklines=true,
}
\newmintedfile[nginxcode]{nginx}{
fontfamily=tt,
linenos=true,
numberblanklines=true,
numbersep=5pt,
gobble=0,
frame=leftline,
framerule=0.4pt,
framesep=2mm,
funcnamehighlighting=true,
tabsize=4,
obeytabs=false,
mathescape=false
samepage=false, %with this setting you can force the list to appear on the same page
showspaces=false,
showtabs =false,
texcl=false,
breaklines=true,
}
% \input{format/code_powershell.tex}

% Ermöglicht das Einbinden langer Quellcode Dateien z. B.
% \begin{longlisting}
%     \yamlcode{code/ansible/playbook.yaml}
%     \caption{Installation des Servers mit Ansible}
%     \label{listing:ansible_playbook}
% \end{longlisting}
\usepackage{caption}
\newenvironment{longlisting}{\captionsetup{type=listing}}{}

% Positionierung von Bilder, s. https://www.overleaf.com/learn/latex/Positioning_images_and_tables#Positioning_images
\usepackage[export]{adjustbox}

%Graf
\usepackage{pgfplots}

% 1.5facher Zeilenabstand.
\linespread{1.5}\selectfont



	
\begin{document}
% Hier wird der Titel-Bereich formatiert
% Die zuvor definierten Textbausteine werden hier verwendet.
% ------------------------------------------------------------------

\newcommand\svthema{Projektdokumentation: Serverhärtung}
\newcommand\svperson{Gruppe 2}
% \newcommand{\svcompany}{Fernuni Inc.} % Nur für einen eventuellen Sperrvermerk
% \newcommand\svmatriculation{123456}
% \newcommand\svcontact{\email{hello@example.com}}
\newcommand\svdatum{\today} % \today für das aktuelle Datum, sonst als Text eintragen
\newcommand\lvname{Fachpraktikum 1599 IT-Sicherheit}
\newcommand\lvtyp{SoSe 2023}
\newcommand\lvinst{Fernuniversität Hagen}
\newcommand\lvprof{Herrn Pascal Tippe}

% Thema und Author in die Meta-Daten der PDF
\hypersetup{ 
    pdftitle={Tätigkeitsbericht \svperson} 
    pdfauthor={\svperson}
}

\title{ \huge\textbf{\svthema} }

% --------------------------------------------------------------------
% Für einen Autor folgendes verwenden:
% --------------------------------------------------------------------
% \author{ \textsl{\svperson} \\ \textsl{\footnotesize\svmatriculation} \\ \textsl{\footnotesize\svcontact}\\
% \\\textsc{"`\lvname"'} \\ {\lvtyp} · {\lvinst} \\ Vorgelegt bei: {\lvprof} \\ Abgabedatum: {\svdatum}}

% \author{ \\ \textsl{\svperson}\\ Matrikelnummer: 3900215  \\ \textsl{\footnotesize\svcontact}\\
% {\lvinst} \\ \\ Abgabedatum:}

% --------------------------------------------------------------------
% Für mehrere Autoren folgendes verwenden:
% --------------------------------------------------------------------
\author{ 
\begin{tabular}{ccc}
Sergej Vormat & Christian Gröbel & Pascal Grundmeier \\
3274268 & 3234371 & 3900215
\end{tabular}}

\date{ \textsc{"`\lvname"'} \\ {\lvtyp} · {\lvinst} \\ Vorgelegt bei: {\lvprof} \\ Abgabedatum: {\svdatum} }

\maketitle
\thispagestyle{empty} % lässt die Seitennummer auf der Titelseite verschwinden
\pagenumbering{Roman} % Römische Seitenzahlen für die Verzeichnisse

\begin{abstract}
    Dieses Dokument dient zur Dokumentation der Serverhärtung der Webserver des vServers der \enquote{Fernuni Inc.}. Für die Umsetzung der Aufgabenstellung sind dabei verschiedenste Aufgaben zu lösen, darunter der Betrieb eines Webservers für verschiedene Domains, die Installation eines Datenbankservers sowie die Absicherung des Systems.
    
    Inhalt des Berichts sind die verschiedenen Aufgabengebiete, die für das Fachpraktikum relevant sind.
\end{abstract}

\cleardoublepage

\tableofcontents			% Inhaltsverzeichnis ... nicht für kurze Dokumente!
\cleardoublepage

% Wichtig, muss enthalten sein! Siehe https://tex.stackexchange.com/a/48511
\phantomsection

% \addcontentsline{toc}{section}{\listfigurename}
% \listoffigures				% Abbildungsverzeichnis ... nicht für kurze Dokumente
% \cleardoublepage
\addcontentsline{toc}{section}{\listtablename}
\listoftables               % Tabellenverzeichnis ... ggf. auskommentieren
% \cleardoublepage
\section*{Abkürzungsverzeichnis}
\addcontentsline{toc}{section}{Abkürzungsverzeichnis}

% Wie funktioniert das Abkürzungsverzeichnis?
% Hier \acro{Abkürzung}{Full} eintragen und im Text dann \ac{Abkürzung} eintragen

% Wichtig: Alphabetisch sortieren!
\begin{acronym}[HTTPS]
    \acro{ACL}{Access Control List}
    \acro{AP}{access point}
    \acro{API}{Application Programming Interface}
    \acro{AWS}{Amazon Web Services}
    \acro{BSI}{Bundesamt für Sicherheit in der Informationstechnik}
    \acro{CD}{Continuous Deployment}
    \acro{CI}{Continuous Integration}
    \acro{CVE}{Common Vulnerabilities and Exposures}
    \acro{DNS}{Domain Name System}
    \acro{DSC}{Desired State Configuration}
    \acro{HTML}{Hypertext Markup Language}
    \acro{HTTP}{Hypertext Transfer Protocol}
    \acro{HTTPS}{Hypertext Transfer Protocol Secure}
    \acro{IaC}{Infrastructure as Code}
    \acro{IDS}{Intrusion Detection System}
    \acro{IP}{Internet Protocol}
    \acro{ITIL}{Information Technology Infrastructure Library}
    \acro{JSON}{JavaScript Object Notation}
    \acro{MFA}{Multi Faktor Authentisierung}
    \acro{MIT}{Massachusetts Institute of Technology}
    \acro{NLTK}{Natural Language Toolkit}
    \acro{NSA}{National Security Agency}
    \acro{PFS}{Perfect Forward Secrecy}
    \acro{QR}{Quick Response}
    \acro{REST}{Representational State Transfer}
    \acro{RHEL}{Red Hat Enterprise Linux}
    \acro{RSS}{Rich Site Summary}
    \acro{SAN}{Storage Attached Network}
    \acro{SQL}{Structured Query Language}
    \acro{SSH}{Secure Shell}
    \acro{TCP}{Transmission Control Protocol}
    \acro{TLS}{Transport Layer Security}
    \acro{URL}{Uniform Resource Locator}
    \acro{VLAN}{Virtual Local Area Network}
    \acro{VPN}{Virtual Private Network}
    \acro{2FA}{Zwei-Faktor-Authentisierung}
\end{acronym}
\printacronyms
   % Abkürzungsverzeichnis ... ggf. auskommentieren

% Quellcode Listings, ggf auskommentieren!
\renewcommand\listoflistingscaption{Liste des enthaltenen Quellcodes}
\listoflistings

\cleardoublepage

\pagenumbering{arabic}
\setcounter{page}{1}

% Vorschlag: Für jedes Kapitel eine eigene .tex Datei anlegen, dann ist die Aufteilung leichter.
\section{Organisatorischer Überblick}

Die Organisation unserer Gruppe erfolgte lediglich online. Besprechungen erfolgten via Discord, kurze Abstimmungen beziehungsweise Statusmeldungen in einer eigens eingerichteten Whatsapp Gruppe.

% Kanban Board
Für die Organisation und Verteilung sämtlicher Aufgaben kam ein eigens dafür eingerichtetes Kanban Board der quelloffenen Kollaborations Plattform \enquote{CryptPad} zum Einsatz. Mithilfe des Boards konnte jeder nachvollziehen welche Aufgaben bereits bearbeitet werden oder noch zu erledigen sind und sich selbstständig eine Aufgabe aussuchen.

% Quellcode erläutern
Des Weiteren wurden im Rahmen der vorliegenden Arbeit sowohl der Quellcode der Dokumentation als auch sämtliche Konfigurationsdateien in einem Git Repository auf der Quellcode Hosting Plattform Github abgelegt. Dadurch wurde sowohl sämtlichen Gruppenmitgliedern uneingeschränkter Zugang als auch eine Versionierung ermöglicht.
\newpage
\section{Analyse potenzieller Bedrohungen}

\underline{Hinweis:} Das hier sind nur Notizen!

\subsection*{Aufgabenstellung}
\begin{itemize}
    \item Bedenken Sie, die IT-Sicherheit umfassend zu betrachten
    \item  Entscheidungen sollen nachvollziehbar dargestellt (insbesondere unter Betrachtung der IT-Sicherheit) werden
    \item Der Fokus liegt nicht auf der konkreten technischen Implementierung, sondern auf der Analyse
    und Vergleich von Technologien
    \item Stellen Sie Überlegungen, Vorgehensweisen und Recherchen dar
\end{itemize}

\subsection{Diskussion potenzieller Angriffsvektoren}

Als Grundlage zur Analyse vermeintlicher Schwachstellen in der Infrastruktur der Fernuni Inc. dient das sogenannte \enquote{Intrusion Kill Chain} Modell des amerikanischen Unternehmens Lockheed Martin.

Dieses Modell unterteilt Cyberangriffe in unterschiedliche Phasen, in denen Angrifer versuchen, IT-Infrastruktur zu kompromittieren. Das Modell basiert dabei auf Informationen, sogenannten Indikatoren, die als Anhaltspunkt für einen Angriff dienen\footcite[Vgl.][]{hutchinsIntelligenceDrivenComputerNetwork}.
% Literatur:
% \url{https://www.lockheedmartin.com/en-us/capabilities/cyber/cyber-kill-chain.html}

% \url{https://de.wikipedia.org/wiki/Cyber_Kill_Chain}
\newpage
\section{Implementierungsphase}

\subsection{Auswahl und Installation der einzusetzenden Software}

\subsubsection{Auswahl eines geeigneten Webserver}

Zur Auswahl eines Webservers muss dieser mindestens die folgenden Anforderungen erfüllen:

\begin{itemize}
    \item Unterstützung für die vom \ac{BSI} empfohlenen \ac{TLS} Versionen 1.2 und 1.3\footcite[Vgl.][S. 6 ff., S. 13 ff.]{bsi}.
    \item Support des Let's Encrypt Certbot für die Ausstellung von Webserver Zertifikaten und automatischen Verlängerung.
    \item Möglichkeit zum Einsatz als Reverse Proxy, um Verbindungen für die Python Applikationen zu terminieren.
    \item Idealerweise gehärtet ab Werk oder Unterstützung zur einfachen Härtung der Konfiguration.
    \item Verfügbarkeit einer hochwertigen Dokumentation.
\end{itemize}

% Allgemeiner Überblick
Zur Auswahl stehen verschiedenste Webserver wie z. B. Apache2, Nginx sowie Caddy. Sämtliche dieser Webserver stehen als quelloffene Software zur Verfügung.

% Überblick Caddy
Caddy unterstützt als einziger der genannten Webserver nicht den Certbot für die automatische Ausstellung von Webserver Zertifikaten und bringt eine eigene Lösung mit. Caddy unterstützt ab Werk \ac{TLS} 1.2 und 1.3, arbeitet als Reverse Proxy und das jüngste Webserver Projekt, welches auf der Programmiersprache Go basiert.

% Apache
Der Webserver Apache2 hingegen existiert bereits seit dem Jahr 1995 und ist hauptsächlich in der Programmiersprache C programmiert. Apache wird von Certbot vollständig unterstützt und kann ebenfalls als Reverse Proxy genutzt werden.

% Nginx
Nginx ist von allen drei Webserver der am meisten verbreitete und kommt auch im Cloud Umfeld z. B. als Ingress Controller für Kubernetes vor. Nginx ist ebenfalls in C programmiert und unterstützt die geforderten \ac{TLS} Versionen und verfügt über eine hochwertige Dokumentation sowie Unterstützung des Certbot.

% Finale Bemerkungen
Sämtliche Webserver verfügen über eine hochwertige Dokumentation und unterstützen die geforderten Anforderungen. Aufgrund der weiten Verbreitung und Tools zur Härtung setzen für auf Nginx als Reverse Proxy.

% Implementierung und Härtung
Die eigentliche Installation des Webservers erfolgt über das im Anhang dargestellte Ansible Playbook (siehe \autoref{listing:ansible_playbook}).

% Härtung
Bei der Härtung sind sowohl die Anforderungen hinsichtlich der kryptografischen Verfahren zu beachten als auch die eigentliche Härtung des Webservers, da dieser von Außen erreichbar ist.

% SSL Config:
% https://ssl-config.mozilla.org/#server=nginx&version=1.17.7&config=intermediate&openssl=1.1.1k&guideline=5.7

% Nginx Server config:
% https://www.digitalocean.com/community/tools/nginx?domains.0.server.domain=messages.fernuni&domains.0.php.php=false&domains.0.reverseProxy.reverseProxy=true&domains.1.server.domain=hello.fernuni&domains.1.php.php=false&domains.1.reverseProxy.reverseProxy=true&domains.1.reverseProxy.proxyPass=http%3A%2F%2F127.0.0.1%3A3001&domains.1.routing.root=false&global.app.lang=de


\begin{listing}
    \nginxcode[firstline=19, lastline=23]{code/nginx/sites-enabled/hello.fernuni.conf}
    \caption{\enquote{hello.fernuni} Nginx Konfiguration}
    \label{listing:nginx_config_hello}
\end{listing}

Der relevante Teil der Nginx Konfiguration für das Reverse Proxying wird in \autoref{listing:nginx_config_hello} dargestellt. Diese Konfiguration sorgt dafür, dass beim Aufruf der Domain \url{https://hello.fernuni} die eingehende Anfrage auf den auf Localhost lauschenden Docker Container auf den \ac{TCP} Port 3001 weitergeleitet wird. Ein identisches Verfahren wird für die Domain \url{https://messages.fernuni} angewendet. Der Einsatz eines Reverse Proxys ermöglicht die TLS Terminierung und sorgt dafür, dass ein Webserver Zertifikat lediglich im Webserver einzurichten ist. Dies verhindert jedoch eine vollständige Ende zu Ende Verschlüsselung.

\subsubsection*{ Aufsetzen einer Datenbank auf Ubuntu}

Die Webseite \enquote{messages.fernuni} erfordert eine Datenbank. Zur Erfüllung dieser Aufgabe wird eine MySQL-Datenbank aufgesetzt. Dabei werden die erforderlichen Installationsschritte für MySQL, Python-Pakete und die Konfiguration der Datenbank beschrieben und durchgeführt.
Die Installation von MySQL erfolgt über den Paketmanager von Ubuntu. Zunächst wird der MySQL-Server installiert, gefolgt von den erforderlichen Python-Paketen für die Datenbankverbindung. Die benötigten Befehle sind folgende:

\begin{minted}{shell}
    sudo apt install mysql-server python3-pip libmariadb-dev
    sudo pip install mysql-connector mysql-connector-python mariadb
\end{minted}

Nach der Installation kann der MySQL-Server mit folgendem Befehl gestartet werden:

\mint{shell}|sudo service mysql start|

Der Status des MySQL-Servers kann mit folgendem Befehl abgerufen werden:

\mint{shell}|sudo service mysql status|

Nachdem der MySQL-Server gestartet ist, wird eine Datenbank und ein Benutzerkonto auf dem MySQL-Server eingerichtet. Die benötigten Informationen für das Erstellen der Datenbank und des Benutzers werden dem Quellcode der \enquote{messages.fernuni.py} entnommen.
Um eine Datenbank zu erstellen und einen Benutzer anzulegen, wird der \verb+MySQL-Client+ verwendet. Dieser kann mit folgendem Befehl aufgerufen werden:
\mint{shell}|sudo mysql -u root|

Anschließend befinden wir uns in der MySQL-Shell. Hier führen wir die folgenden Befehle aus, um die Datenbank und den Benutzer anzulegen:

\begin{minted}{sql}
    CREATE DATABASE messages;
    CREATE USER 'testuser'@'localhost' IDENTIFIED BY '58tqxEJxqf123';
    GRANT ALL PRIVILEGES ON messages.* TO 'testuser'@'localhost';
\end{minted}

Um eine Tabelle in der Datenbank zu erstellen, verwenden wir erneut die \verb+MySQL Shell+.
Als Erstes wechseln wir zur Datenbank \enquote{messages} und erstellen dort die Tabelle \enquote{messagestorage} mit der Spalte \enquote{timestamp}:

\begin{minted}{sql}
    USE messages;
    CREATE TABLE messagestorage (message TEXT);
    ALTER TABLE messagestorage ADD COLUMN timestamp TIMESTAMP;
\end{minted}

Falls erforderlich, kann der MySQL-Server mit dem folgenden Befehl neu gestartet werden:

\mint{shell}|sudo service mysql restart|

Nun ist die Datenbank, die für \enquote{messages.fernuni} erforderlich ist, aufgesetzt und die Funktionalität der Webseite gewährleistet.

Erstellen eines automatischen Backups der MySQL-Datenbank.

Das automatische Backup gewährleistet den Schutz wichtiger Daten und ermöglicht die Wiederherstellung im Falle eines Datenverlusts oder eines Systemfehlers. Die folgenden Schritte erläutern die Einrichtung des automatischen Backups für unsere Datenbank.
Erstellen einer Skriptdatei für das Backup.
Durch das Erstellen einer Skriptdatei können wir den Backup-Prozess automatisieren und die Ausführungsschritte zentralisieren.
Im Verzeichnis \enquote{/etc/mysql} erstellt man eine Skriptdatei namens \verb+mysql_backup.sh+ mit dem folgenden Inhalt:

\begin{minted}[breaklines=true]{shell}
    #!/bin/bash
    sudo mysqldump -u root messages > /etc/mysql/backup/backup_messages.sql
\end{minted}

Dieses Skript verwendet den Befehl \enquote{mysqldump}, um eine Sicherungskopie der Datenbank \enquote{messages} zu erstellen. 
Die Backup-Datei wird im Verzeichnis /etc/mysql/backup/ mit dem Namen \verb+backup_messages.sql+ gespeichert.

Ausführbarkeit der Skriptdatei gewährleisten

Um die Skriptdatei ausführbar zu machen, gibt man den folgenden Befehl ein:
\begin{verbatim} sudo chmod +x mysql_backup.sh \end{verbatim}
Dadurch wird die Ausführungsberechtigung für die Skriptdatei aktiviert.
Ausführen des Skripts im Terminal, um sicherzustellen, dass es ordnungsgemäß funktioniert:
\mint{shell}|sudo ./mysql_backup.sh|

Einrichten eines regelmäßigen automatischen Updates.

Dazu gibt man den folgenden Befehl ein, um den Crontab-Editor zu öffnen:
\mint{shell}|sudo crontab -e|
Die Verwendung des Crontab-Dienstes ermöglicht es uns, den \verb+Backup-Job+ regelmäßig auszuführen. Durch das Hinzufügen einer Zeile zur Crontab geben wir an, wann und wie oft das Backup-Skript ausgeführt werden soll. 

\begin{verbatim}0 18 * * * /etc/mysql/mysql_backup.sh\end{verbatim}

Diese Zeile gibt die gewünschte Zeit für das Backup an. Hier wird das Backup Skript täglich um 18:00 Uhr ausgeführt.


% \begin{listing}[ht]{}
%     \shellcode{code/shell/webserver_installieren.sh}
%     \caption{Installation des Apache Webservers}
%     \label{listing:installation_apache}
% \end{listing}

\subsubsection{Umsetzung der Mandatory Access Control mittels AppArmor}

Mittels AppArmor ist die Zuweisung von Profilen zu einer Anwendung möglich. Ein Profil besteht dabei aus Regeln, die z. B. lesenden oder schreibenden Zugriff zu einer Datei oder einem Verzeichnis oder sogar Netzwerkzugang ermöglichen. Im Falle einer Kompromittierung kann AppArmor somit des Gesamtsystem schützen.

Im Fokus der Betrachtung stehen insbesondere Anwendungen mit Netzwerkzugriff, da diese auch aus der Ferne angreifbar sind. Im Falle einer Rechteausweitung mittels Privilege Escalation begrenzt AppArmor den potenziellen Schaden\cite{hutchinsIntelligenceDrivenComputerNetwork}.

Dies betrifft insbesondere die Dienste \ac{SSH}, \ac{HTTP} bzw. \ac{HTTPS} sowie Wireguard. Auf Github existieren mehrere Projekte, die fertige Profile für AppArmor anbieten, insbesondere das \enquote{apparmor.d}-Projekt, welches nach eigenen Angaben etwa 1400 Profile bereitstellt. Dabei werden auch Dienste erfasst, welche unter dem \enquote{root} Benutzer ausgeführt werden und folglich besonders anfällig für eine Rechteausweitung sind.

% Begründen, warum ausgerechnet apparmor.d
Das \enquote{apparmor.d}-Projekt eignet sich am ehesten zur Umsetzung umfassender AppArmor Profile, weil dies sowohl aktiv weiterentwickelt wird als auch über eine umfassende Dokumentation verfügt. Es existieren zwar noch andere Projekt, dies ist nach der Anzahl der Beitragenden sowie Commits am weitesten entwickelt.

% Installation kurz beschreiben
Die konkrete Installation der AppArmor Profile ist im Anhang in \autoref{listing:installation_apparmor} beschrieben. In der Dokumentation des Projekts wird explizit darauf hingewiesen, zunächst die Lauffähigkeit des Systems zuprüfen, bevor der Enforce Mode aktiviert.

% Auf Tests eingehen.
Die Überprüfung möglicher Probleme bei der Durchsetzung der Profile erfolgt über den Befehl \mint{shell}|sudo aa-log|
und zeigt dabei sowohl blockierte als auch erlaubte Aktionen an, wie in der folgenden Ausgabe anhand des Docker Daemons erkennbar ist:

\begin{minted}[breaklines=true]{shell}
vagrant@ubuntu-jammy:~$ sudo aa-log 
ALLOWED dockerd mount /var/lib/docker/check-overlayfs-support2956447284/merged/ info="failed mntpnt match" comm=dockerd fstype=overlay srcname=overlay error=-13
ALLOWED dockerd umount /var/lib/docker/check-overlayfs-support2956447284/merged/ comm=dockerd
\end{minted}

\subsubsection{Konfiguration der lokalen Firewall}


\begin{table}[!ht]
    \centering
    \begin{adjustbox}{width=\textwidth}

    \begin{tabular}{|l|l|l|l|l|l|l|l|l|}
        \hline
            Nr. & Protokoll & Quell-IP & Quell-Port & Ziel-IP & Ziel-Port & Interface & -m State & Aktion \\ \hline
            1 & TCP & Internal & ANY & 192.168.2.80 & 22 & eth0 & NEW,ESTABLISHED & ALLOW \\ \hline
            2 & TCP & ANY & ANY & 192.168.2.80 & 80/443 & eth0 & NEW,ESTABLISHED & ALLOW \\ \hline
            3 & TCP & ANY & ANY & 192.168.2.80 & Wireguard & eth0 & NEW,ESTABLISHED & ALLOW \\ \hline
        \end{tabular}
    \end{adjustbox}
    \caption{Firewall Tabelle für eingehenden Verkehr}
    \label{regeln_fw_incoming}
\end{table}
% Tabelle erstellen: \url{https://tableconvert.com/latex-generator}

Eine MySQL Regel entfällt, weil lediglich vom lokalen System Verbindungen hergestellt werden. Andernfalls sollte dies ebenfalls auf nur erforderliche Adressen eingerichtet werden.

Zugriff via \ac{SSH} ist lediglich aus dem internen Netzwerk (\enquote{Internal}) gestattet, sodass Zugriff aus dem öffentlichen Internet in gänze unterbunden werden, wie dies auch das Cyber-Killchain Modell vorsieht.


% --------------------------------------------------------------------
% Hier die entsprechende .bib-Datei eintragen
% --------------------------------------------------------------------
% !BIB program = biblatex
% \addbibresource{literature/seminar-lit.bib}

% Warnung vermeiden
% \BiblatexSplitbibDefernumbersWarningOff

% Ggf. heading= usw. entfernen, um aus dem Inhaltsverzeichnis zu entfernen.
% \printbibliography[nottype=online, heading=bibintoc]
% \printbibliography[type=online, heading=bibintoc, title={Internetquellen}]


% \cleardoublepage
% \input{ee.tex}
\end{document}