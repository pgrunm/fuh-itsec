\documentclass[12pt,a4paper]{scrartcl}

% Grundlegende Einstellungen
\usepackage[ngerman]{babel}			% deutsche Namen/Umlaute
\usepackage[utf8]{inputenc}			% Zeichensatzkodierung
\usepackage{graphicx}				% Einbinden von Bildern
\usepackage{color}					% Farben wenn es sein muß
\usepackage[hidelinks]{hyperref}    % Klickbare Verweise und \autoref{label}
\providecommand*{\listingautorefname}{Listing}
\usepackage{bookmark}
\newcommand*{\email}[1]{            % Klickbare eMail-Adressen
    \normalsize\href{mailto:#1}{#1}\par
}
\usepackage{booktabs}				% "Schöne" Tabellen
\usepackage{amsmath}				% Mathematischer Formelsatz AMS
\usepackage{amsfonts}
\usepackage{csquotes}               % Empfohlener Import für babel/polyglossia
\usepackage{blindtext}              % Einfügen von Blindtext möglich

\usepackage[
    printonlyused                   % nur verwendete Abkürzungen ins Verzeichnis
]{acronym}                          % Abkürzungen
\usepackage[section]{placeins}      % Bilder bleiben im Abschnitt
% \usepackage{scrpage2}
% \usepackage{scrlayer-scrpage}       % Neuere Version von scrpage2
\usepackage[automark,headsepline,singlespacing=true]{scrlayer-scrpage}
\pagestyle{scrheadings}

% Neuer Befehl
% \clearmainofpairofpagestyles

% Aufzählungen anpassen wie hier beschrieben: https://www.latex-tutorial.com/tutorials/lists/
\usepackage{enumitem}

% Abkürzungen
\usepackage[acronym]{glossaries}

% Silbentrennung wie hier beschrieben: https://tex.stackexchange.com/a/119514
\usepackage[T1]{fontenc}

% Quellcode für Python
\usepackage{minted}
\newmintedfile[shellcode]{shell}{
fontfamily=tt,
linenos=true,
numberblanklines=true,
numbersep=5pt,
gobble=0,
frame=leftline,
framerule=0.4pt,
framesep=2mm,
funcnamehighlighting=true,
tabsize=4,
obeytabs=false,
mathescape=false
samepage=false, %with this setting you can force the list to appear on the same page
showspaces=false,
showtabs =false,
texcl=false,
}
\newmintedfile[pythoncode]{python}{
fontfamily=tt,
linenos=true,
numberblanklines=true,
numbersep=5pt,
gobble=0,
frame=leftline,
framerule=0.4pt,
framesep=2mm,
funcnamehighlighting=true,
tabsize=4,
obeytabs=false,
mathescape=false
samepage=false, %with this setting you can force the list to appear on the same page
showspaces=false,
showtabs =false,
texcl=false,
}
\newmintedfile[yamlcode]{yaml}{
fontfamily=tt,
linenos=true,
numberblanklines=true,
numbersep=5pt,
gobble=0,
frame=leftline,
framerule=0.4pt,
framesep=2mm,
funcnamehighlighting=true,
tabsize=4,
obeytabs=false,
mathescape=false
samepage=false, %with this setting you can force the list to appear on the same page
showspaces=false,
showtabs =false,
texcl=false,
breaklines=true,
}
\newmintedfile[nginxcode]{nginx}{
fontfamily=tt,
linenos=true,
numberblanklines=true,
numbersep=5pt,
gobble=0,
frame=leftline,
framerule=0.4pt,
framesep=2mm,
funcnamehighlighting=true,
tabsize=4,
obeytabs=false,
mathescape=false
samepage=false, %with this setting you can force the list to appear on the same page
showspaces=false,
showtabs =false,
texcl=false,
breaklines=true,
}
% \input{format/code_powershell.tex}

% Ermöglicht das Einbinden langer Quellcode Dateien z. B.
% \begin{longlisting}
%     \yamlcode{code/ansible/playbook.yaml}
%     \caption{Installation des Servers mit Ansible}
%     \label{listing:ansible_playbook}
% \end{longlisting}
\usepackage{caption}
\newenvironment{longlisting}{\captionsetup{type=listing}}{}

% Positionierung von Bilder, s. https://www.overleaf.com/learn/latex/Positioning_images_and_tables#Positioning_images
\usepackage[export]{adjustbox}

%Graf
\usepackage{pgfplots}

% 1.5facher Zeilenabstand.
\linespread{1.5}\selectfont



	
\begin{document}
% Hier wird der Titel-Bereich formatiert
% Die zuvor definierten Textbausteine werden hier verwendet.
% ------------------------------------------------------------------

\newcommand\svthema{Projektdokumentation: Serverhärtung}
\newcommand\svperson{Gruppe 2}
% \newcommand{\svcompany}{Fernuni Inc.} % Nur für einen eventuellen Sperrvermerk
% \newcommand\svmatriculation{123456}
% \newcommand\svcontact{\email{hello@example.com}}
\newcommand\svdatum{\today} % \today für das aktuelle Datum, sonst als Text eintragen
\newcommand\lvname{Fachpraktikum 1599 IT-Sicherheit}
\newcommand\lvtyp{SoSe 2023}
\newcommand\lvinst{Fernuniversität Hagen}
\newcommand\lvprof{Herrn Pascal Tippe}

% Thema und Author in die Meta-Daten der PDF
\hypersetup{ 
    pdftitle={Tätigkeitsbericht \svperson} 
    pdfauthor={\svperson}
}

\title{ \huge\textbf{\svthema} }

% --------------------------------------------------------------------
% Für einen Autor folgendes verwenden:
% --------------------------------------------------------------------
% \author{ \textsl{\svperson} \\ \textsl{\footnotesize\svmatriculation} \\ \textsl{\footnotesize\svcontact}\\
% \\\textsc{"`\lvname"'} \\ {\lvtyp} · {\lvinst} \\ Vorgelegt bei: {\lvprof} \\ Abgabedatum: {\svdatum}}

% \author{ \\ \textsl{\svperson}\\ Matrikelnummer: 3900215  \\ \textsl{\footnotesize\svcontact}\\
% {\lvinst} \\ \\ Abgabedatum:}

% --------------------------------------------------------------------
% Für mehrere Autoren folgendes verwenden:
% --------------------------------------------------------------------
\author{ 
\begin{tabular}{ccc}
Sergej Vormat & Christian Gröbel & Pascal Grundmeier \\
3274268 & 3234371 & 3900215
\end{tabular}}

\date{ \textsc{"`\lvname"'} \\ {\lvtyp} · {\lvinst} \\ Vorgelegt bei: {\lvprof} \\ Abgabedatum: {\svdatum} }

\maketitle
\thispagestyle{empty} % lässt die Seitennummer auf der Titelseite verschwinden
\pagenumbering{Roman} % Römische Seitenzahlen für die Verzeichnisse

\begin{abstract}
    Dieses Dokument dient zur Dokumentation der Serverhärtung der Webserver des vServers der \enquote{Fernuni Inc.}. Für die Umsetzung der Aufgabenstellung sind dabei verschiedenste Aufgaben zu lösen, darunter der Betrieb eines Webservers für verschiedene Domains, die Installation eines Datenbankservers sowie die Absicherung des Systems.
    
    Inhalt des Berichts sind die verschiedenen Aufgabengebiete, die für das Fachpraktikum relevant sind.
\end{abstract}

\cleardoublepage

\tableofcontents			% Inhaltsverzeichnis ... nicht für kurze Dokumente!
\cleardoublepage

% Wichtig, muss enthalten sein! Siehe https://tex.stackexchange.com/a/48511
\phantomsection

% \addcontentsline{toc}{section}{\listfigurename}
% \listoffigures				% Abbildungsverzeichnis ... nicht für kurze Dokumente
% \cleardoublepage
\addcontentsline{toc}{section}{\listtablename}
\listoftables               % Tabellenverzeichnis ... ggf. auskommentieren
% \cleardoublepage
\section*{Abkürzungsverzeichnis}
\addcontentsline{toc}{section}{Abkürzungsverzeichnis}

% Wie funktioniert das Abkürzungsverzeichnis?
% Hier \acro{Abkürzung}{Full} eintragen und im Text dann \ac{Abkürzung} eintragen

% Wichtig: Alphabetisch sortieren!
\begin{acronym}[HTTPS]
    \acro{ACL}{Access Control List}
    \acro{AP}{access point}
    \acro{API}{Application Programming Interface}
    \acro{AWS}{Amazon Web Services}
    \acro{BSI}{Bundesamt für Sicherheit in der Informationstechnik}
    \acro{CD}{Continuous Deployment}
    \acro{CI}{Continuous Integration}
    \acro{CVE}{Common Vulnerabilities and Exposures}
    \acro{DNS}{Domain Name System}
    \acro{DSC}{Desired State Configuration}
    \acro{HTML}{Hypertext Markup Language}
    \acro{HTTP}{Hypertext Transfer Protocol}
    \acro{HTTPS}{Hypertext Transfer Protocol Secure}
    \acro{IaC}{Infrastructure as Code}
    \acro{IDS}{Intrusion Detection System}
    \acro{IP}{Internet Protocol}
    \acro{ITIL}{Information Technology Infrastructure Library}
    \acro{JSON}{JavaScript Object Notation}
    \acro{MFA}{Multi Faktor Authentisierung}
    \acro{MIT}{Massachusetts Institute of Technology}
    \acro{NLTK}{Natural Language Toolkit}
    \acro{NSA}{National Security Agency}
    \acro{PFS}{Perfect Forward Secrecy}
    \acro{QR}{Quick Response}
    \acro{REST}{Representational State Transfer}
    \acro{RHEL}{Red Hat Enterprise Linux}
    \acro{RSS}{Rich Site Summary}
    \acro{SAN}{Storage Attached Network}
    \acro{SQL}{Structured Query Language}
    \acro{SSH}{Secure Shell}
    \acro{TCP}{Transmission Control Protocol}
    \acro{TLS}{Transport Layer Security}
    \acro{URL}{Uniform Resource Locator}
    \acro{VLAN}{Virtual Local Area Network}
    \acro{VPN}{Virtual Private Network}
    \acro{2FA}{Zwei-Faktor-Authentisierung}
\end{acronym}
\printacronyms
   % Abkürzungsverzeichnis ... ggf. auskommentieren

% Quellcode Listings, ggf auskommentieren!
% \renewcommand\listoflistingscaption{Liste des enthaltenen Quellcodes}
% \listoflistings

\cleardoublepage

\pagenumbering{arabic}
\setcounter{page}{1}

% Vorschlag: Für jedes Kapitel eine eigene .tex Datei anlegen, dann ist die Aufteilung leichter.
\section{Hello World}

TODO: Seite mit gutem Content füllen.

\newpage

\subsection{Hello World Subchapter}

TODO: Seite mit gutem Content füllen.



% --------------------------------------------------------------------
% Hier die entsprechende .bib-Datei eintragen
% --------------------------------------------------------------------
% !BIB program = biblatex
% \addbibresource{literature/seminar-lit.bib}

% Warnung vermeiden
% \BiblatexSplitbibDefernumbersWarningOff

% Ggf. heading= usw. entfernen, um aus dem Inhaltsverzeichnis zu entfernen.
% \printbibliography[nottype=online, heading=bibintoc]
% \printbibliography[type=online, heading=bibintoc, title={Internetquellen}]


% \cleardoublepage
% \input{ee.tex}
\end{document}