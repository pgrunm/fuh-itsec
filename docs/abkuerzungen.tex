\section*{Abkürzungsverzeichnis}
\addcontentsline{toc}{section}{Abkürzungsverzeichnis}

% Wie funktioniert das Abkürzungsverzeichnis?
% Hier \acro{Abkürzung}{Full} eintragen und im Text dann \ac{Abkürzung} eintragen

% Wichtig: Alphabetisch sortieren!
\begin{acronym}[HTTPS]
    \acro{ACL}{Access Control List}
    \acro{AP}{access point}
    \acro{API}{Application Programming Interface}
    \acro{AWS}{Amazon Web Services}
    \acro{BSI}{Bundesamt für Sicherheit in der Informationstechnik}
    \acro{CD}{Continuous Deployment}
    \acro{CI}{Continuous Integration}
    \acro{CVE}{Common Vulnerabilities and Exposures}
    \acro{DNS}{Domain Name System}
    \acro{DSC}{Desired State Configuration}
    \acro{HTML}{Hypertext Markup Language}
    \acro{HTTP}{Hypertext Transfer Protocol}
    \acro{HTTPS}{Hypertext Transfer Protocol Secure}
    \acro{IaC}{Infrastructure as Code}
    \acro{IDS}{Intrusion Detection System}
    \acro{ITIL}{Information Technology Infrastructure Library}
    \acro{JSON}{JavaScript Object Notation}
    \acro{MFA}{Multi Faktor Authentisierung}
    \acro{MIT}{Massachusetts Institute of Technology}
    \acro{NLTK}{Natural Language Toolkit}
    \acro{NSA}{National Security Agency}
    \acro{PFS}{Perfect Forward Secrecy}
    \acro{QR}{Quick Response}
    \acro{REST}{Representational State Transfer}
    \acro{RHEL}{Red Hat Enterprise Linux}
    \acro{RSS}{Rich Site Summary}
    \acro{SAN}{Storage Attached Network}
    \acro{SQL}{Structured Query Language}
    \acro{SSH}{Secure Shell}
    \acro{TCP}{Transmission Control Protocol}
    \acro{TLS}{Transport Layer Security}
    \acro{URL}{Uniform Resource Locator}
    \acro{VLAN}{Virtual Local Area Network}
\end{acronym}
\printacronyms
