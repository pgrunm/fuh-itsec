\usepackage[ngerman]{babel}			% deutsche Namen/Umlaute
\usepackage[utf8]{inputenc}			% Zeichensatzkodierung
\usepackage{graphicx}				% Einbinden von Bildern
\usepackage{color}					% Farben wenn es sein muß
\usepackage[hidelinks]{hyperref}    % Klickbare Verweise und \autoref{label}
\usepackage{bookmark}
\newcommand*{\email}[1]{            % Klickbare eMail-Adressen
    \normalsize\href{mailto:#1}{#1}\par
}
\usepackage{booktabs}				% "Schöne" Tabellen
\usepackage{amsmath}				% Mathematischer Formelsatz AMS
\usepackage{amsfonts}
\usepackage{csquotes}               % Empfohlener Import für babel/polyglossia
\usepackage{blindtext}              % Einfügen von Blindtext möglich

\usepackage[
    printonlyused                   % nur verwendete Abkürzungen ins Verzeichnis
]{acronym}                          % Abkürzungen
\usepackage[section]{placeins}      % Bilder bleiben im Abschnitt
% \usepackage{scrpage2}
% \usepackage{scrlayer-scrpage}       % Neuere Version von scrpage2
\usepackage[automark,headsepline,singlespacing=true]{scrlayer-scrpage}
\pagestyle{scrheadings}

% Neuer Befehl
% \clearmainofpairofpagestyles

% Aufzählungen anpassen wie hier beschrieben: https://www.latex-tutorial.com/tutorials/lists/
\usepackage{enumitem}

% Abkürzungen
\usepackage[acronym]{glossaries}

% Silbentrennung wie hier beschrieben: https://tex.stackexchange.com/a/119514
\usepackage[T1]{fontenc}

% Quellcode für Python
\usepackage{minted}
\newmintedfile[shellcode]{shell}{
fontfamily=tt,
linenos=true,
numberblanklines=true,
numbersep=5pt,
gobble=0,
frame=leftline,
framerule=0.4pt,
framesep=2mm,
funcnamehighlighting=true,
tabsize=4,
obeytabs=false,
mathescape=false
samepage=false, %with this setting you can force the list to appear on the same page
showspaces=false,
showtabs =false,
texcl=false,
}
\newmintedfile[pythoncode]{python}{
fontfamily=tt,
linenos=true,
numberblanklines=true,
numbersep=5pt,
gobble=0,
frame=leftline,
framerule=0.4pt,
framesep=2mm,
funcnamehighlighting=true,
tabsize=4,
obeytabs=false,
mathescape=false
samepage=false, %with this setting you can force the list to appear on the same page
showspaces=false,
showtabs =false,
texcl=false,
}
% \input{format/code_powershell.tex}

% Positionierung von Bilder, s. https://www.overleaf.com/learn/latex/Positioning_images_and_tables#Positioning_images
\usepackage[export]{adjustbox}

%Graf
\usepackage{pgfplots}

% 1.5facher Zeilenabstand.
\linespread{1.5}\selectfont

