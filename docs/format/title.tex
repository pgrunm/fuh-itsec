% ------------------------------------------------------------------

\newcommand\svthema{Projektdokumentation: Serverhärtung}
\newcommand\svperson{Gruppe 2}
% \newcommand{\svcompany}{Fernuni Inc.} % Nur für einen eventuellen Sperrvermerk
% \newcommand\svmatriculation{123456}
% \newcommand\svcontact{\email{hello@example.com}}
\newcommand\svdatum{\today} % \today für das aktuelle Datum, sonst als Text eintragen
\newcommand\lvname{Fachpraktikum 1599 IT-Sicherheit}
\newcommand\lvtyp{SoSe 2023}
\newcommand\lvinst{Fernuniversität Hagen}
\newcommand\lvprof{Herrn Pascal Tippe}

% Thema und Author in die Meta-Daten der PDF
\hypersetup{ 
    pdftitle={Tätigkeitsbericht \svperson} 
    pdfauthor={\svperson}
}

\title{ \huge\textbf{\svthema} }

% --------------------------------------------------------------------
% Für einen Autor folgendes verwenden:
% --------------------------------------------------------------------
% \author{ \textsl{\svperson} \\ \textsl{\footnotesize\svmatriculation} \\ \textsl{\footnotesize\svcontact}\\
% \\\textsc{"`\lvname"'} \\ {\lvtyp} · {\lvinst} \\ Vorgelegt bei: {\lvprof} \\ Abgabedatum: {\svdatum}}

% \author{ \\ \textsl{\svperson}\\ Matrikelnummer: 3900215  \\ \textsl{\footnotesize\svcontact}\\
% {\lvinst} \\ \\ Abgabedatum:}

% --------------------------------------------------------------------
% Für mehrere Autoren folgendes verwenden:
% --------------------------------------------------------------------
\author{ 
\begin{tabular}{ccc}
Sergej Vormat & Christian Gröbel & Pascal Grundmeier \\
123456 & 654321 & 98765\\
\email{hello@web.de} & \email{hello@web.de} & \email{hello@web.de}
\end{tabular}}

\date{ \textsc{"`\lvname"'} \\ {\lvtyp} · {\lvinst} \\ Vorgelegt bei: {\lvprof} \\ Abgabedatum: {\svdatum} }
