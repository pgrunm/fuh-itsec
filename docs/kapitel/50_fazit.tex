\section{Fazit}

\subsection{Bewertung der Praxisnähe der Aufgaben}
% TODO: Bewertung der Netzwerksichereit in der Praxis.
Die Aufgabenstellungen des Fachpraktikums stellen in der Tat praxisnahe Aufgaben dar. So ist die Verteidigung gegen Angriffe von Innen sowie Außen nach wie vor relevant. Dies betrifft insbesondere die Absicherung von sensiblen Diensten wie beispielsweise \ac{SSH}. Wird dieser von einem Angreifer benutzt, hat dieser bereits Zugriff auf das System und kann von dort aus weiter agieren.

% Datenbankzugang
Ebenso sind Datenbanken sowie deren Backups als sensibel zu betrachten, da dort häufig vertrauliche Geschäftsdaten lagern, die sowohl für Konkurrenten als auch Angreifer wertvoll sind. Durch den umfassenden Einsatz von Verschlüsselung schließt man dieses Risiko gänzlich aus und beugt zudem noch juristischen Problemen durch Sanktionsmaßnahmen der europäischen Datenschutzgrundverordnung vor.

% SSHGuard und IDS erläutern
Zusätzlich zur eigentlichen Härtung vorhandener Dienste wie \ac{SSH} bietet sich die Installation weiterer Sicherheitssoftware wie SSHGuard oder ein \ac{IDS} an. SSHGuard einerseits verhindert Brute Force Angriffe und bannt temporär die bösartige IP Adresse.

\subsection{Retrospektive des Fachpraktikums}
Im Rahmen der Aufgaben des Fachpraktikums hat sich gezeigt, dass IT-Sicherheit keine einfach abzuschließende Aufgabe darstellt. Es gibt immer Möglichkeiten ein System, eine Anwendung oder Dienst noch sicherer zu konfigurieren. 

Die Möglichkeiten sind quasi unbegrenzt und stellen immer eine Abwägung zwischen Nutzen und Aufwand dar. Häufig existiert zudem nicht \underline{die} eine Quelle der Wahrheit, sodass die Härtung eines Systems einen iterativen und fehleranfälligen Prozess darstellt.

Unsere Gruppe war sehr heterogen zusammengesetzt, sodass wir von den unterschiedlichen Erfahrungen jedes Mitglieds profitieren konnten. 