\section{Implementierung der Sicherheitsmaßnahmen}

\subsection{Installation der ausgewählten Software}

\subsubsection{Apache als Webserver}

Installation eines Webservers für zwei Webseiten auf Ubuntu System.
Der Webserver soll zwei separate Webseiten, \enquote{hello.fernuni} und \enquote{messages.fernuni}, bedienen. Beide Webseiten sind in der Programmiersprache Python geschrieben. Die Dokumentation erläutert die Gründe hinter jedem Schritt und gibt detaillierte Anweisungen zur Durchführung der Installation.

Als Team 3 haben wir uns für den Apache Webserver entschieden, denn Apache ist ein weit verbreiteter und zuverlässiger Webserver, der auf vielen Plattformen eingesetzt wird. Die Installation erfolgt mithilfe des Paketmanagers von Ubuntu.

Apache installieren.
Um den Apache-Webserver zu installieren, führen wir folgenden Befehl aus:
sudo apt-get install apache2

Flask installieren
Um Flask, ein Python-Framework für Webanwendungen, zu installieren, verwenden wir den folgenden Befehl:
sudo apt install python3-flask
Flask ermöglicht die Entwicklung von Webanwendungen mit Python. In diesem Fall wird Flask verwendet, um die Webseiten \enquote{hello.fernuni} und \enquote{messages.fernuni} zu implementieren.

Hostdatei ändern
Die Hostdatei muss angepasst werden, um die lokal gehosteten Webseiten zu erreichen. Die \enquote{hosts}-Datei befindet sich in /etc/hosts. Wir öffenen die \enquote{hosts}-Datei mit einem Texteditor (z. B. Nano oder Vim) und fügen die folgenden Zeilen hinzu:
127.0.0.1 	hello.fernuni
127.0.0.1 	messages.fernuni
Durch das Hinzufügen der entsprechenden Einträge in der \enquote{hosts}-Datei kann der Webbrowser die lokal gehosteten Webseiten unter den angegebenen Domains finden.

Proxy-Modul aktivieren
Das Proxy-Modul ermöglicht die Weiterleitung von HTTP-Anfragen an andere Webserver. Es wird benötigt, um die Anfragen für die Webseiten \enquote{hello.fernuni} und \enquote{messages.fernuni} an die entsprechenden Python-Server weiterzuleiten.
Das Proxy-Modul in Apache wird mit den folgenden Befehlen aktiviert:
sudo a2enmod proxy
sudo a2enmod proxy_html
sudo a2enmod proxy_http

Modul mod_wsgi installieren
Das mod_wsgi-Modul ermöglicht die Ausführung von Python-Anwendungen innerhalb des Apache-Webservers. Es wird verwendet, um die Flask-Anwendungen für die Webseiten zu integrieren.
Wir installieren das Python-Paket mod_wsgi mit dem folgenden Befehl:
sudo apt-get install libapache2-mod-wsgi-py3


Python-Dateien ablegen
Apache verwendet standardmäßig das Verzeichnis \enquote{/var/www/html} als den Wurzelordner für gehostete Webinhalte. Durch das Ablegen der Python-Dateien in diesem Verzeichnis können sie vom Webserver erkannt und ausgeführt werden.
Aus diesem Grund legen wir die Python-Dateien für die Webseiten \enquote{hello.fernuni} und \enquote{messages.fernuni} im Verzeichnis \enquote{/var/www/html/hello}  bzw. \enquote{/var/www/html/messages} ab. 

Konfigurationsdateien erstellen
Der Webserver benötigt für das bedienen von Webseiten Konfigurationsdateien.
Durch das Erstellen von separaten Konfigurationsdateien für jede Webseite können die spezifischen Einstellungen und Proxy-Weiterleitungen für jede Seite festgelegt werden.
Dafür erstellen wir für jede Webseite eine separate Konfigurationsdatei im Verzeichnis \enquote{/etc/apache2/sites-enabled}. 
Dabei hat die Datei \enquote{hello.fernuni.config} folgenden Inhalt:
<VirtualHost *:80> 
    ProxyPreserveHost On
    ProxyRequest Off
    ServerName hello.fernuni 
    ServerAlias www.hello.fernuni 
    DocumentRoot /var/www/html/hello
    ProxyPass / http://localhost:8080
    ProxyPassReverse / http://localhost:8080
    ErrorLog ${APACHE_LOG_DIR}/hello.fernuni_error.log
    CustomLog ${APACHE_LOG_DIR}/hello.fernuni_access.log combined
</VirtualHost>

Analog hat die Datei \enquote{messages.fernuni.config} folgenden Inhalt:
<VirtualHost *:80> 
    ProxyPreserveHost On
    ProxyRequest Off
    ServerName messages.fernuni 
    ServerAlias www.messages.fernuni 
    DocumentRoot /var/www/html/messages
    ProxyPass / http://localhost:7500
    ProxyPassReverse / http://localhost:7500
    ErrorLog ${APACHE_LOG_DIR}/messages.fernuni_error.log
    CustomLog ${APACHE_LOG_DIR}/messagesfernuni_access.log combined
</VirtualHost>

Virtuelle Hosts aktivieren
Durch die Aktivierung der virtuellen Hosts werden die Konfigurationsdateien mit dem Verzeichnis \enquote{/etc/apache2/sites-enabled} verknüpft. Dadurch kann Apache die entsprechenden Webseiten bedienen.
Wir aktivieren die virtuellen Hosts, indem wir die Befehle \enquote{a2ensite} für jede Konfigurationsdatei ausführen.
sudo a2ensite hello.fernuni.config
sudo a2ensite messages.fernuni.config

Apache-Webserver neu starten
Um die Konfigurationsänderungen zu übernehmen, muss der Apache-Webserver neugestartet werden:
sudo service apache2 restart

Starten der Python-Server
Damit die Webseiten \enquote{hello.fernuni} und \enquote{messages.fernuni} durch den Apache-Webserver bedienbar sind, müssen diese gestartet werden. Dies kann man mit folgenden Befehlen in den jeweiligen Ordnern erreichen:
python3 hello.fernuni.py
python3 messages.fernuni.py


\begin{listing}[ht]{}
    \shellcode{code/shell/webserver_installieren.sh}
    \caption{Installation des Apache Webservers}
    \label{listing:installation_apache}
\end{listing}

\subsubsection{Umsetzung der Mandatory Access Control mittels AppArmor}

Mittels AppArmor ist die Zuweisung von Profilen zu einer Anwendung möglich. Ein Profil besteht dabei aus Regeln, die z. B. lesenden oder schreibenden Zugriff zu einer Datei oder einem Verzeichnis oder sogar Netzwerkzugang ermöglichen. Im Falle einer Kompromittierung kann AppArmor somit des Gesamtsystem schützen.

Im Fokus der Betrachtung stehen insbesondere Anwendungen mit Netzwerkzugriff, da diese auch aus der Ferne angreifbar sind. Im Falle einer Rechteausweitung mittels Privilege Escalation begrenzt AppArmor den potenziellen Schaden\cite{hutchinsIntelligenceDrivenComputerNetwork}.

\subsubsection{Konfiguration der lokalen Firewall}


\begin{table}[!ht]
    \centering
    \begin{adjustbox}{width=\textwidth}

    \begin{tabular}{|l|l|l|l|l|l|l|l|l|}
        \hline
            Nr. & Protokoll & Quell-IP & Quell-Port & Ziel-IP & Ziel-Port & Interface & -m State & Aktion \\ \hline
            1 & TCP & Internal & ANY & 192.168.2.80 & 22 & eth0 & NEW,ESTABLISHED & ALLOW \\ \hline
            2 & TCP & ANY & ANY & 192.168.2.80 & 80/443 & eth0 & NEW,ESTABLISHED & ALLOW \\ \hline
            3 & TCP & ANY & ANY & 192.168.2.80 & Wireguard & eth0 & NEW,ESTABLISHED & ALLOW \\ \hline
        \end{tabular}
    \end{adjustbox}
    \caption{Firewall Tabelle für eingehenden Verkehr}
    \label{regeln_fw_incoming}
\end{table}
Tabelle erstellen: \url{https://tableconvert.com/latex-generator}

\subsubsection{Härtung des Systems}

% Literatur
% https://github.com/trimstray/the-practical-linux-hardening-guide#policy-compliance
% https://dev-sec.io/

\subsubsection{Einrichtung eines Intrusion Detection Systems}

OSSEC, Einrichtung via Ansible.
