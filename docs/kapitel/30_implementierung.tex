\section{Implementierungsphase}

\subsection{Auswahl und Installation der einzusetzenden Software}

\subsubsection{Auswahl eines geeigneten Webservers}

Zur Auswahl eines Webservers muss dieser mindestens die folgenden Anforderungen erfüllen:

\begin{itemize}
    \item Unterstützung für die vom \ac{BSI} empfohlenen \ac{TLS} Versionen 1.2 und 1.3\footcite[Vgl.][S. 6 ff., S. 13 ff.]{bsi}.
    \item Support des Let's Encrypt Certbot für die Ausstellung von Webserver Zertifikaten und automatischen Verlängerung.
    \item Möglichkeit zum Einsatz als Reverse Proxy, um Verbindungen für die Python-Applikationen zu terminieren.
    \item Idealerweise gehärtet ab Werk oder Unterstützung zur einfachen Härtung der Konfiguration.
    \item Verfügbarkeit einer hochwertigen Dokumentation.
\end{itemize}

% Allgemeiner Überblick
Zur Auswahl stehen zahlreiche Webserver wie z. B. Apache2, Nginx sowie Caddy. Sämtliche dieser Webserver stehen als quelloffene Software zur Verfügung.

% Überblick Caddy
Caddy unterstützt als einziger der genannten Webserver nicht den Certbot für die automatische Ausstellung von Webserver Zertifikaten und bringt eine eigene Lösung mit. Caddy unterstützt ab Werk \ac{TLS} 1.2 und \ac{TLS} 1.3, arbeitet als Reverse Proxy und das jüngste Webserver Projekt, welches auf der Programmiersprache Go basiert.

% Apache
Der Webserver Apache2 existiert hingegen bereits seit dem Jahr 1995 und ist hauptsächlich in der Programmiersprache C programmiert. Apache wird von Certbot vollständig unterstützt und kann ebenfalls als Reverse Proxy genutzt werden.

% Nginx
Nginx ist von allen drei Webservern der am meisten verbreitete und kommt auch im Cloud Umfeld z. B. als Ingress Controller für Kubernetes vor. Nginx ist ebenfalls in C programmiert, unterstützt die geforderten \ac{TLS} Versionen und verfügt über eine hochwertige Dokumentation sowie Unterstützung des Certbot\footcite[Vgl.][]{certbot}.

% Finale Bemerkungen
Sämtliche Webserver verfügen über eine hochwertige Dokumentation und unterstützen die geforderten Anforderungen. Aufgrund der weiten Verbreitung und Tools zur Härtung setzen die Autoren auf Nginx als Reverse Proxy.

Allerdings ist es erforderlich, dass der Nginx Master Prozess als Root Benutzer läuft, denn nur auf diese Weise ist es möglich, auf den Port 80 bzw. 443 zu lauschen. Diese werden als priviligierte Ports betrachtet und sind standardmäßig lediglich für administrative Benutzer verfügbar.

% Implementierung und Härtung
Die eigentliche Installation des Webservers erfolgt über das im Anhang dargestellte Ansible Playbook (siehe \autoref{listing:ansible_playbook}).

Damit der Webserver auch die Domains \url{messages.fernuni} und \url{hello.fernuni} auflösen kann, muss dies im \ac*{DNS} eingetragen werden. Im \ac{DNS} müsste dies z. B. wie folgt aufgelöst werden:

\begin{minted}[breaklines=true]{shell}
q hello.fernuni
hello.fernuni. 1h0m0s A 132.176.108.129
q messages.fernuni
messages.fernuni. 1h0m0s A 132.176.108.129
\end{minted}

Durch diese Einträge werden Anfragen an die IP-Adresse des Webservers geschickt, die dann von Nginx als Reverse Proxy beantwortet werden. Dies kann in der Nginx Konfiguration in \autoref{kap:nginx} detailliert nachgesehen werden.

% Härtung
Bei der Härtung sind sowohl die Anforderungen hinsichtlich der kryptografischen Verfahren zu beachten als auch die eigentliche Härtung des Webservers, da dieser von außen erreichbar ist.

% SSL Config:
% https://ssl-config.mozilla.org/#server=nginx&version=1.17.7&config=intermediate&openssl=1.1.1k&guideline=5.7

% Nginx Server config:
% https://www.digitalocean.com/community/tools/nginx?domains.0.server.domain=messages.fernuni&domains.0.php.php=false&domains.0.reverseProxy.reverseProxy=true&domains.1.server.domain=hello.fernuni&domains.1.php.php=false&domains.1.reverseProxy.reverseProxy=true&domains.1.reverseProxy.proxyPass=http%3A%2F%2F127.0.0.1%3A3001&domains.1.routing.root=false&global.app.lang=de


\begin{listing}[ht]
    \nginxcode[firstline=19, lastline=23]{code/nginx/sites-enabled/hello.fernuni.conf}
    \caption{\enquote{hello.fernuni} Nginx Konfiguration}
    \label{listing:nginx_config_hello}
\end{listing}

Der relevante Teil der Nginx Konfiguration für das Reverse Proxying wird in \autoref{listing:nginx_config_hello} dargestellt. Diese Konfiguration sorgt dafür, dass beim Aufruf der Domain \url{https://hello.fernuni} die eingehende Anfrage auf den auf Localhost lauschenden Docker Container auf den \ac{TCP} Port 3001 weitergeleitet wird. Ein identisches Verfahren wird für die Domain \url{https://messages.fernuni} angewendet. Der Einsatz eines Reverse Proxys ermöglicht die TLS Terminierung und sorgt dafür, dass ein Webserver Zertifikat lediglich im Webserver einzurichten ist. Dies verhindert jedoch eine vollständige Ende-zu-Ende-Verschlüsselung. Die vollständige Konfiguration des Nginx ist im Anhang in \autoref{kap:nginx} ersichtlich.

Da die vollständige Ende-zu-Ende-Verschlüsselung nicht gewährleistet werden kann, gibt es mehrere Risiken im Zusammenhang mit der Sicherheit. So besteht die Möglichkeit zum Abhören sensibler Daten. Die Angreifer können sensible Informationen wie Benutzernamen, Passwörter, Kreditkarteninformationen oder andere vertrauliche Daten aus dem Datenverkehr extrahieren.

Zudem besteht die Möglichkeit eines sogenannten Man-in-the-Middle-Angriffs. Hierbei kann ein Angreifer den Datenverkehr abfangen, manipulieren oder sogar gefälschte Antworten an den Nginx-Server senden, um den Nutzer zu täuschen oder sensible Informationen zu stehlen. Beispielsweise können die Angreifer schädlichen Code in die übertragenen Daten einschleusen. Dies kann dazu führen, dass bösartiger Code auf dem Server oder im Container ausgeführt wird, was zu Sicherheitslücken oder Datenverlust führen kann.

Nicht zu unterschätzen ist das Risiko zur Verletzung der Compliance-Anforderungen. Viele Branchen und Vorschriften erfordern den Schutz von personenbezogenen Daten und anderen vertraulichen Informationen. Durch die Weiterleitung unverschlüsselter Datenverbindungen an Container oder Server könnten diese Compliance-Anforderungen verletzt werden. Eine solche Verletzung kann zu erheblichen rechtlichen Konsequenzen und Strafen führen. 

\subsubsection{Aufsetzen einer Datenbank auf Ubuntu}

Die Webseite \enquote{messages.fernuni} erfordert eine Datenbank. Zur Erfüllung dieser Aufgabe wird eine MySQL-Datenbank aufgesetzt. Dabei werden die erforderlichen Installationsschritte für MySQL, Python-Pakete und die Konfiguration der Datenbank beschrieben und durchgeführt.

% Begründung warum nicht wir keine Container nutzen
Anders als z. B. die bereitgestellten Python Applikationen handelt es bei einer Datenbank nicht um eine zustandslose Anwendung. Bei Containern handelt es sich meist um zustandslose Anwendungen wie Webserver o. ä., die beliebig skalierbar sind. Durch die Containerisierung wird zudem eine zusätzliche Isolationsschicht zwischen dem Host und dem Container implementiert. Durch diese zusätzliche Isolation steigt der Overhead bei der Verwaltung der Datenbank und es werden mehr Ressourcen zur Verwaltung benötigt. Aus diesen Gründen findet keine Containerisierung der Datenbank statt. Bei Bedarf könnten stattdessen die Docker Container auf einen separaten Server oder sogar via Kubernetes bereitgestellt werden, sodass auf dem vServer lediglich das Datenbankmanagementsystem installiert ist.

Die Installation von MySQL erfolgt über den Paketmanager von Ubuntu. Zunächst wird der MySQL-Server installiert, gefolgt von den erforderlichen Python-Paketen für die Datenbankverbindung. Die benötigten Befehle sind folgende:

\begin{minted}{shell}
sudo apt install mysql-server python3-pip libmariadb-dev
sudo pip install mysql-connector mysql-connector-python mariadb
\end{minted}

Nach der Installation kann der MySQL-Server mit folgendem Befehl gestartet werden:

\mint{shell}|sudo service mysql start|

Der Status des MySQL-Servers kann mit folgendem Befehl abgerufen werden:

\mint{shell}|sudo service mysql status|

Nachdem der MySQL-Server gestartet ist, wird eine Datenbank und ein Benutzerkonto auf dem MySQL-Server eingerichtet. Die benötigten Informationen für das Erstellen der Datenbank und des Benutzers werden dem Quellcode der \enquote{messages.fernuni.py} entnommen.
Um eine Datenbank zu erstellen und einen Benutzer anzulegen, wird der \verb+MySQL-Client+ verwendet. Dieser kann mit folgendem Befehl aufgerufen werden:
\mint{shell}|sudo mysql -u root|

Anschließend befinden wir uns in der MySQL-Shell. Hier führen wir die folgenden Befehle aus, um die Datenbank und den Benutzer anzulegen:

\begin{minted}{sql}
    CREATE DATABASE messages;
    CREATE USER 'testuser'@'localhost' IDENTIFIED BY '58tqxEJxqf123';
    GRANT ALL PRIVILEGES ON messages.* TO 'testuser'@'localhost';
\end{minted}

Um eine Tabelle in der Datenbank zu erstellen, verwenden wir erneut die \verb+MySQL Shell+.
Als Erstes wechseln wir zur Datenbank \enquote{messages} und erstellen dort die Tabelle \enquote{messagestorage} mit der Spalte \enquote{timestamp}:

\begin{minted}{sql}
    USE messages;
    CREATE TABLE messagestorage (message TEXT);
    ALTER TABLE messagestorage ADD COLUMN timestamp TIMESTAMP;
\end{minted}

Falls erforderlich kann der MySQL-Server mit dem folgenden Befehl neu gestartet werden:

\mint{shell}|sudo service mysql restart|

Nun ist die Datenbank, die für \enquote{messages.fernuni} erforderlich ist, aufgesetzt und die Funktionalität der Webseite gewährleistet.

\subsubsection*{Erstellen eines automatischen Backups der MySQL-Datenbank}

Das automatische Backup gewährleistet den Schutz wichtiger Daten und ermöglicht die Wiederherstellung im Falle eines Datenverlusts oder eines Systemfehlers. Die folgenden Schritte erläutern die Einrichtung des automatischen Backups für unsere Datenbank.

\subsubsection*{Erstellen einer Skriptdatei für das Backup}

Durch das Erstellen einer Skriptdatei können wir den Backup-Prozess automatisieren und die Ausführungsschritte zentralisieren.
Im Verzeichnis \enquote{/etc/mysql} wird eine Skriptdatei namens \verb+mysql_backup.sh+ mit dem folgenden Inhalt erstellt:

\begin{listing}
    
    \begin{minted}[fontfamily=tt,
        linenos=true,
        numberblanklines=true,
        numbersep=5pt,
        gobble=0,
        frame=leftline,
        framerule=0.4pt,
        framesep=2mm,
        funcnamehighlighting=true,
        tabsize=4,
        obeytabs=false,
        mathescape=false
    samepage=false, %with this setting you can force the list to appear on the same page
    showspaces=false,
    showtabs =false,
    texcl=false,
    breaklines=true]{shell}
    #!/bin/bash
    # Create a MySQL dump of the messages database, zip and encrypt it.
    file_name=$(date +%Y-%m-%d-%H.%M.%S).sql.bz2.enc
    sudo mysqldump -u root messages | bzip2 -c | age -e -R /etc/mysql/backup/encryption_key.txt -o /etc/mysql/backup/$file_name
    # Copy the file to remote
    rsync -a /etc/mysql/backup/$file_name backup_user@backup_server:/var/backups/mysql/server01/$file_name
\end{minted}
\caption{MySQL Datenbankbackup Skript}
\end{listing}

Dieses Skript verwendet den Befehl \enquote{mysqldump}, um eine Sicherungskopie der Datenbank \enquote{messages} zu erstellen. 
Die Backup-Datei wird im Verzeichnis /etc/mysql/backup/ mit dem Namen \verb+backup_messages.sql+ gespeichert.

% Backup erläutern.
Zusätzlich wird das Datenbankbackup komprimiert und mittels asymmetrischer Verschlüsselung durch die Verschlüsselungssoftware \enquote{age} verschlüsselt. Da es sich um eine asymmetrische Verschlüsselung handelt, muss der öffentliche Schlüssel nicht geschützt werden und kann lokal auf dem virtuellen Server liegen. Die gesicherte und verschlüsselte Datei wird im Anschluss noch auf einen Remote Server kopiert, auf dem sie anschließend archiviert wird.

\subsubsection*{Ausführbarkeit der Skriptdatei gewährleisten}

Um die Skriptdatei ausführbar zu machen, wird der folgende Befehl benötigt:
\begin{verbatim}sudo chmod +x mysql_backup.sh \end{verbatim}
Dadurch wird die Ausführungsberechtigung für die Skriptdatei aktiviert.
Die Ausführung des Skripts im Terminal, um sicherzustellen, dass es ordnungsgemäß funktioniert:
\mint{shell}|sudo ./mysql_backup.sh|

\subsubsection*{Einrichten eines regelmäßigen automatischen Backups}

Dazu gibt man den folgenden Befehl ein, um den Crontab-Editor zu öffnen:
\mint{shell}|sudo crontab -e|
Die Verwendung des Crontab-Dienstes ermöglicht es uns, den \verb+Backup-Job+ regelmäßig auszuführen. Durch das Hinzufügen einer Zeile zur Crontab geben wir an, wann und wie oft das Backup-Skript ausgeführt werden soll. 

\begin{verbatim}*/15 * * * * /etc/mysql/mysql_backup.sh\end{verbatim}

Diese Zeile gibt die gewünschte Zeit für das Backup an. Hier wird das Backup Skript alle 15 Minuten ausgeführt, sodass im schlimmsten Fall maximal 15 Minuten an Daten fehlen. Bei Bedarf kann das Intervall auch verringert werden, da die Installation per Ansible Playbook erfolgt.

\subsubsection{Containerisierung der bereitgestellten Programme}

Eine der gestellten Anforderungen der \enquote{Fernuni Inc.} stellt die Containerisierung der in Python implementierten Programme \enquote{hello.fernuni.py} und \enquote{messages.fernuni.py} dar.

Bei Python handelt es sich um eine interpretierte Programmiersprache, sodass eine Laufzeitumgebung zur Ausführung erforderlich ist. Diese muss zusätzlich mit in das Container Image installiert werden, damit das Programm lauffähig ist.

Eine Überlegung zur Härtung der Docker Images wäre die Verwendung eines sogenannten \enquote{Distroless Image}, dass ohne ein Betriebssystem bereitgestellt wird. Allerdings wird vom Einsatz eines solchen Images für Python im Produktivbetrieb bislang abgeraten, sodass dies im Kritis-Umfeld nicht möglich ist.

% Kompilierung erwähnen
Eine Idee der Autoren, stellte die Kompilierung des vorhandenen Quellcodes in eine native ausführbare Datei dar, sodass lediglich diese Datei in einen Distroless Image zu kopieren wäre. Einige Versuche mit dem experimentellen Compiler \href{https://github.com/exaloop/codon}{Codon} schlugen leider fehl, aufgrund fehlender Unterstützung für weitere Module wie beispielsweise Flask. Leider existiert bislang kein ausgereifter Python Compiler, der dies unterstützt, sodass diese Möglichkeit verworfen wurde.

Stattdessen wird beim Bau des \enquote{Messages} Images auf ein sogenanntes \enquote{Multi-Stage-Build} Verfahren gesetzt. Im Rahmen dieses Prozesses wird zunächst ein Container für die Installation sämtlicher benötigter Pakete und zusätzlicher Software zur Installation gestartet. Im Anschluss werden die benötigten Dateien in ein noch leeres Image kopiert und dort als ausführender User \enquote{App} festgelegt. Dies verhindert, dass der Container mit Root-Rechten läuft. Aufgrund dieser weiteren Abhängigkeiten für das Mariadb Modul wird in diesem Fall das größere \enquote{Python Slim} Image als Basis genutzt. Dies enthält alle benötigten Softwarebibliotheken zur Installation.

\begin{longlisting}
    \begin{minted}[fontfamily=tt,
        linenos=true,
        numberblanklines=true,
        numbersep=5pt,
        gobble=0,
        frame=leftline,
        framerule=0.4pt,
        framesep=2mm,
        funcnamehighlighting=true,
        tabsize=4,
        obeytabs=false,
        mathescape=false
        samepage=false, %with this setting you can force the list to appear on the same page
        showspaces=false,
        showtabs =false,
        texcl=false,
    breaklines=true]{docker}
# Build image, used to build everything we need
FROM python:3.11-slim as builder

WORKDIR /usr/src/app

# Look for any new updates, install them and install wget.
RUN apt update && apt install -y wget gcc

# Install Mariadb connector, as it's required.
RUN wget https://dlm.mariadb.com/2862620/Connectors/c/connector-c-3.3.4/mariadb-connector-c-3.3.4-debian-bullseye-amd64.tar.gz -O - | tar -zxf - --strip-components=1 -C /usr

# Create a virtual Python environment
RUN python -m venv /usr/src/app/venv
ENV PATH="/usr/src/app/venv/bin:$PATH"

RUN pip install --upgrade pip; pip install --no-cache-dir flask mariadb mysql-connector-python
COPY . .

FROM python:3.11-slim

RUN groupadd -r app && useradd -r -g app app app

WORKDIR /usr/src/app
COPY --from=builder /usr/src/app /usr/src/app
COPY --from=builder /usr /usr

USER app
ENV PATH="/usr/src/app/venv/bin:$PATH"

# Listen on TCP port 8080
EXPOSE 8080
CMD [ "python", "./messages.fernuni.py" ]
\end{minted}
\caption{Dockerfile Messages Programm}
\end{longlisting}

Für den Bau des Containers nutzten die Autoren u. a. insbesondere einen Eintrag zur Installation eines neueren MySQL Konnektors\footcite[Vgl.][]{Danblack_answer_2022} sowie eines Artikels von Multi Stage Images\footcite[Vgl.][]{turner-trauring_multi-stage_2019}.

% Härtung erläutern
Das Image für das Messages Programm hat noch externe Abhängigkeiten zu einer C-Bibliothek (MySQL Konnektor) und benötigt zum Download z. B. das Programm \enquote{wget}. Die Installation auf einem Distroless System war nicht möglich, sodass weiterhin das Python Slim Image genutzt wird.

Für die Bereitstellung des \enquote{Hello} Container Image werden keinerlei externe Abhängigkeiten außer Python benötigt. Deshalb dient das Distroless Python Image als Basis für das Container Image. Dort wird lediglich Flask sowie das Python Skript installiert sowie die Ausführung unter einem nicht Root Benutzer eingerichtet. Weiterhin wird bei der Installation der Python-Pakete via Pip auf ein Cache Verzichnis verzichtet, damit das finale Image möglichst klein ist. Das Distroless Image verzichtet in Gänze auf ein Betriebssystem oder einen Paketmanager und steigert die Sicherheit immens.

Anders als übliche Docker Basis Images zeichnet sich dieses Image zusätzlich dadurch aus, dass es täglich neu erzeugt wird und in regelmäßigen Image Scans über keinerlei bekannte Sicherheitslücken, sogenannte \ac{CVE}, verfügt.

\subsubsection{Umsetzung der Mandatory Access Control mittels AppArmor}\label{kap:installation_apparmor}

Mittels AppArmor ist die Zuweisung von Profilen zu einer Anwendung möglich. Ein Profil besteht dabei aus Regeln, die z. B. lesenden oder schreibenden Zugriff zu einer Datei oder einem Verzeichnis oder sogar Netzwerkzugang ermöglichen. Im Falle einer Kompromittierung kann AppArmor somit des Gesamtsystem schützen.

Im Fokus der Betrachtung stehen insbesondere Anwendungen mit Netzwerkzugriff, da diese auch aus der Ferne angreifbar sind. Im Falle einer Rechteausweitung mittels Privilege Escalation begrenzt AppArmor den potenziellen Schaden\footcite[Vgl.][]{hutchinsIntelligenceDrivenComputerNetwork}.

Dies betrifft insbesondere die Dienste \ac{SSH}, \ac{HTTP} bzw. \ac{HTTPS} sowie Wireguard. Auf Github existieren mehrere Projekte, die fertige Profile für AppArmor anbieten, insbesondere das \enquote{apparmor.d}-Projekt, welches nach eigenen Angaben etwa 1400 Profile bereitstellt. Dabei werden auch Dienste erfasst, welche unter dem \enquote{root} Benutzer ausgeführt werden und folglich besonders anfällig für eine Rechteausweitung sind.

% Begründen, warum ausgerechnet apparmor.d
Das \enquote{apparmor.d}-Projekt eignet sich am ehesten zur Umsetzung umfassender AppArmor Profile, weil dies sowohl aktiv weiterentwickelt wird als auch über eine umfassende Dokumentation verfügt. Es existieren zwar auch andere Projekt, dies ist jedoch nach der Anzahl der Beitragenden sowie Commits am weitesten entwickelt.

% Installation kurz beschreiben
Die konkrete Installation der AppArmor Profile ist im Anhang in \autoref{listing:installation_apparmor} beschrieben. In der Dokumentation des Projekts wird explizit darauf hingewiesen, zunächst die Lauffähigkeit des Systems zu prüfen, bevor der Enforce Mode aktiviert wird.

% Auf Tests eingehen.
Die Überprüfung möglicher Probleme bei der Durchsetzung der Profile erfolgt über den Befehl \mint{shell}|sudo aa-log|
und zeigt dabei sowohl blockierte als auch erlaubte Aktionen an, wie in der folgenden Ausgabe anhand des Docker Daemons erkennbar ist:

\begin{minted}[breaklines=true]{shell}
vagrant@ubuntu-jammy:~$ sudo aa-log 
ALLOWED dockerd mount /var/lib/docker/check-overlayfs-support2956447284/merged/ info="failed mntpnt match" comm=dockerd fstype=overlay srcname=overlay error=-13
ALLOWED dockerd umount /var/lib/docker/check-overlayfs-support2956447284/merged/ comm=dockerd
\end{minted}

Eine ausführliche Ausgabe dieses Befehls ist im Anhang in \autoref{listing:log_apparmor} ersichtlich. Zur Aktivierung des Enforce Mode genügt es, im Verzeichnis \enquote{debian/rules} die folgenden Zeilen im Makefile zu ergänzen:

\begin{minted}{makefile}
override_dh_auto_build:
    make enforce
\end{minted}

Durch diese Änderung ist statt des Complain Mode der Enforce Mode nach der Installation aktiv.

\subsubsection{Konfiguration der lokalen Firewall}

Zur Konfiguration der Firewall wird das Programm \enquote{ufw} genutzt, welches als einfach zu nutzendes Kommandozeilenwerkzeug zur Verwaltung der Netfilter Firewall unter Linux dient. \enquote{ufw} gilt dabei als einfachere Alternative zu \enquote{iptables} und ist standardmäßig in Ubuntu als Paket installiert.

\begin{table}[!ht]
    \centering
    \begin{adjustbox}{width=\textwidth}

    \begin{tabular}{|l|l|l|l|l|l|l|l|l|}
        \hline
            Nr. & Protokoll & Quell-IP & Quell-Port & Ziel-IP & Ziel-Port & Interface & -m State & Aktion \\ \hline
            1 & TCP & Internal & ANY & 192.168.2.80 & 22 & eth0 & NEW,ESTABLISHED & ALLOW \\ \hline
            2 & TCP & ANY & ANY & 192.168.2.80 & 80/443 & eth0 & NEW,ESTABLISHED & ALLOW \\ \hline
            3 & TCP & ANY & ANY & 192.168.2.80 & Wireguard & eth0 & NEW,ESTABLISHED & ALLOW \\ \hline
        \end{tabular}
    \end{adjustbox}
    \caption{Übersicht Firewall Regeln}
    \label{regeln_fw_incoming}
\end{table}
% Tabelle erstellen: \url{https://tableconvert.com/latex-generator}

Eine MySQL Regel entfällt, weil lediglich vom lokalen System Verbindungen hergestellt werden. Andernfalls sollte dies ebenfalls auf nur erforderliche Adressen eingerichtet werden. MySQL lauscht in der Standardeinstellung lediglich auf Localhost und externe Verbindungen sind in der derzeitigen Konfiguration nicht vorgesehen. Dementsprechend ist dies folglich als die sicherste Variante zum Schutz des Datenbankservers anzusehen.

Zugriff via \ac{SSH} ist lediglich aus dem internen Netzwerk (\enquote{Internal}, was hier als Platzhalter dient) gestattet, sodass Zugriff aus dem öffentlichen Internet in Gänze unterbunden werden, wie dies auch das Cyber-Killchain Modell vorsieht.

% https://abuse.ch/#platforms
Das Projekt \href{https://abuse.ch}{Abuse.ch} bietet eine regelmäßig aktualisierte Liste mit bösartigen IP-Adressen an, die in Verbindung mit der Verbreitung von Malware oder Botnets stehen. Folglich bietet es sich an, diese IP-Adressen zusätzlich in der Firewall zu blockieren, um weitere Angriffe einzuschränken.

Das im Anhang unter \autoref{kap:fw_abuse} dargestellte Skript ruft die IP-Adressen ab, löscht zunächst sämtliche vorher eingetragenen Adressen und fügt dann neue Regeln in die lokale Firewall ein. Somit wird verhindert, dass der Server diese IP-Adressen kontaktieren kann und die Kompromittierung durch bösartige Akteure weiter verringert.
