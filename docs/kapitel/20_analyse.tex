\section{Analyse potenzieller Bedrohungen}

\underline{Hinweis:} Das hier sind nur Notizen!

\subsection*{Aufgabenstellung}
\begin{itemize}
    \item Bedenken Sie, die IT-Sicherheit umfassend zu betrachten
    \item  Entscheidungen sollen nachvollziehbar dargestellt (insbesondere unter Betrachtung der IT-Sicherheit) werden
    \item Der Fokus liegt nicht auf der konkreten technischen Implementierung, sondern auf der Analyse
    und Vergleich von Technologien
    \item Stellen Sie Überlegungen, Vorgehensweisen und Recherchen dar
\end{itemize}

\subsection{Diskussion potenzieller Angriffsvektoren}

Als Grundlage zur Analyse vermeintlicher Schwachstellen in der Infrastruktur der Fernuni Inc. dient das sogenannte \enquote{Intrusion Kill Chain}-Modell des amerikanischen Unternehmens Lockheed Martin.

Dieses Modell unterteilt Cyberangriffe in unterschiedliche Phasen, in denen Angrifer versuchen, IT-Infrastruktur zu kompromittieren. Das Modell basiert dabei auf Informationen, sogenannten Indikatoren, die als Anhaltspunkt für einen Angriff dienen\footcite[Vgl.][]{hutchinsIntelligenceDrivenComputerNetwork}.

Das Modell listet dabei auch Maßnahmen auf, mit denen Angriffe in den verschiedenen Phasen aufgespürt, verhindert, gestört, abgeschwächt oder getäuscht werden können.

\begin{table}[]
    \begin{center}
        \begin{tabular}{|l|l|l|}
            \hline
            Phase                & Detect        & Deny         \\ \hline
            Reconnaissance       & Web analytics & Firewall     \\ \hline
            Weaponization        & NIDS          & NIPS         \\ \hline
            Delivery             & Vigilant user & Proxy filter \\ \hline
            Exploitation         & HIDS          & Patch        \\ \hline
            Installation         & HIDS          &              \\ \hline
            C2                   & NIDS          & Firewall     \\ \hline
    Actions on Objective & Audit log     &              \\ \hline
        \end{tabular}
    \caption{Liste mit Werkzeugen zur Detektion oder Verhinderung von Angriffen in verschiedenen Phasen}
    \label{table:defense_tools}
    \end{center}
\end{table}
% Literatur:
% \url{https://www.lockheedmartin.com/en-us/capabilities/cyber/cyber-kill-chain.html}

% \url{https://de.wikipedia.org/wiki/Cyber_Kill_Chain}