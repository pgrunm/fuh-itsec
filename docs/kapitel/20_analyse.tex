\section{Diskussion und Analyse potenzieller Angriffsvektoren}

Als Grundlage zur Analyse vermeintlicher Schwachstellen in der Infrastruktur der Fernuni Inc. dient das sogenannte \enquote{Intrusion Kill Chain}-Modell des amerikanischen Unternehmens Lockheed Martin.

Dieses Modell unterteilt Cyberangriffe in unterschiedliche Phasen, in denen Angreifer versuchen, IT-Infrastruktur zu kompromittieren. Das Modell basiert dabei auf Informationen (Indikatoren), die als Anhaltspunkt für einen Angriff dienen\footcite[Vgl.][]{hutchinsIntelligenceDrivenComputerNetwork}.

Das Modell beschreibt dabei auch Maßnahmen auf, mit denen Angriffe in den verschiedenen Phasen aufgespürt, verhindert, gestört, abgeschwächt oder getäuscht werden können. \autoref{table:defense_tools} gibt eine Übersicht über die verschiedenen Phasen sowie potenzielle Werkzeuge.

\begin{table}[ht]
    \begin{center}
        \begin{tabular}{|l|l|l|}
            \hline
            Phase                & Detect        & Deny         \\ \hline
            Reconnaissance       & Web analytics & Firewall     \\ \hline
            Weaponization        & NIDS          & NIPS         \\ \hline
            Delivery             & Vigilant user & Proxy filter \\ \hline
            Exploitation         & HIDS          & Patch        \\ \hline
            Installation         & HIDS          &              \\ \hline
            C2                   & NIDS          & Firewall     \\ \hline
    Actions on Objective & Audit log     &              \\ \hline
        \end{tabular}
    \caption{Liste mit Werkzeugen zur Detektion oder Verhinderung von Angriffen in verschiedenen Phasen}
    \label{table:defense_tools}
    \end{center}
\end{table}

Da nicht sämtliche Phasen aus zeitlichen Gründen im Rahmen des Fachpraktikums umgesetzt werden können, greifen die Autoren im Folgenden lediglich die Phasen auf, die eine besondere Relevanz darstellen. Überdies soll dies dem \enquote{Defense in Depth} Konzept folgen, das für das Erschweren möglicher Angriffe auf die IT-Infrastruktur die Implementierung verschiedener Sicherheitsmaßnahmen vorsieht\footcite[Vgl.][]{barbu2015defense}.

Dazu zählen folgende Maßnahmen, wie z. B. Transportverschlüsselung und der Speicherung, Firewalls, \ac{MFA}, Prüfung der Integrität durch kryptografische Prüfsummen, Logging, sichere Anmeldeverfahren und die Durchführung regelmäßiger Aktualisierungen von installierten Anwendungen.

Ausgehend von \autoref{table:defense_tools} existieren zwei potenzielle Strategien zur Verteidigung gegen Angriffe: Entweder durch die Aufspürung eines Angriffs oder durch die Verhinderung dessen. Die Verhinderung ist nicht in sämtlichen Phasen möglich, die Aufspürung jedoch in sämtlichen Phasen.

Im Rahmen unserer Vorgaben und mit der Orientierung am Killchain Modell, setzen wir zur Verteidigung zusätzlich die \autoref{tab:tools_killchain} abgebildeten Softwares ein.

\begin{table}[ht]
    \begin{center}
        \begin{tabular}{|l|l|l|}
            \hline
            Phase        & Detektion      & Abwehr        \\ \hline
            Aufklärung   &                & \href{https://feodotracker.abuse.ch/blocklist/}{Firewall}            \\ \hline
            Angriff      &                & SSHGuard            \\ \hline
            Übermittlung &                & \href{https://gitlab.com/malware-filter/urlhaus-filter}{URLhaus Filter}     \\ \hline
            Ausnutzung   &                & Unattended Upgrades \\ \hline
            Installation & OSSEC          &                     \\ \hline
            C2           &                & Firewall            \\ \hline
            Manipulation & Auditd Service/OSSEC &                     \\ \hline
        \end{tabular}
    \end{center}
    \caption{Auswahl verschiedener Softwares zur Abwehr von Cyberangriffen}
    \label{tab:tools_killchain}
\end{table}

Die in der \autoref{tab:tools_killchain} aufgelisteten Softwares wie z. B. SSHGuard oder Unattended Upgrades werden im \autoref{kap:haertung_des_systems} näher betrachtet.