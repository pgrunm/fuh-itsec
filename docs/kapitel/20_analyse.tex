\section{Analyse potenzieller Bedrohungen}

\subsection{Diskussion potenzieller Angriffsvektoren}

Als Grundlage zur Analyse vermeintlicher Schwachstellen in der Infrastruktur der Fernuni Inc. dient das sogenannte \enquote{Intrusion Kill Chain}-Modell des amerikanischen Unternehmens Lockheed Martin.

Dieses Modell unterteilt Cyberangriffe in unterschiedliche Phasen, in denen Angrifer versuchen, IT-Infrastruktur zu kompromittieren. Das Modell basiert dabei auf Informationen, sogenannten Indikatoren, die als Anhaltspunkt für einen Angriff dienen\footcite[Vgl.][]{hutchinsIntelligenceDrivenComputerNetwork}.

Das Modell listet dabei auch Maßnahmen auf, mit denen Angriffe in den verschiedenen Phasen aufgespürt, verhindert, gestört, abgeschwächt oder getäuscht werden können. \autoref{table:defense_tools} gibt eine Übersicht über die verschiedenen Phasen sowie potenziell Werkzeuge.

\begin{table}[ht]
    \begin{center}
        \begin{tabular}{|l|l|l|}
            \hline
            Phase                & Detect        & Deny         \\ \hline
            Reconnaissance       & Web analytics & Firewall     \\ \hline
            Weaponization        & NIDS          & NIPS         \\ \hline
            Delivery             & Vigilant user & Proxy filter \\ \hline
            Exploitation         & HIDS          & Patch        \\ \hline
            Installation         & HIDS          &              \\ \hline
            C2                   & NIDS          & Firewall     \\ \hline
    Actions on Objective & Audit log     &              \\ \hline
        \end{tabular}
    \caption{Liste mit Werkzeugen zur Detektion oder Verhinderung von Angriffen in verschiedenen Phasen}
    \label{table:defense_tools}
    \end{center}
\end{table}
% Literatur:
% \url{https://www.lockheedmartin.com/en-us/capabilities/cyber/cyber-kill-chain.html}

% \url{https://de.wikipedia.org/wiki/Cyber_Kill_Chain}

% https://en.wikipedia.org/wiki/Defense_in_depth_(computing)
Nicht sämtliche Phasen können während der kurzen Phase des Fachpraktikums umgesetzt werden, sodass wir uns auf die aus unserer Sicht wichtigsten konzentrieren. Dies soll zudem dem sogenannten Konzept von \enquote{Defense in Depth} folgen.

Dieses Konzept sieht vor viele von einander unabhängige Sicherheitsmaßnahmen zu implementieren, um Angriffe auf IT Infrastruktur möglichst schwierig zu gestalten\footcite[Vgl.][]{barbu2015defense}.

Dazu zählen Maßnahmen wie z. B. Verschlüsselung während des Transports und der Speicherung, Firewalls, \ac{MFA}, Prüfung der Integrität durch kryptografische Prüfsummen, Logging, sichere Anmeldeverfahren und die Durchführung regelmäßiger Aktualisierungen von installierten Anwendungen.

Ausgehend von \autoref{table:defense_tools} existieren zwei potenzielle Strategien, zur Verteidigung gegen Angriffe: Entweder durch die Aufspürung eines Angriffs oder durch die Verhinderung dessen. Die Verhinderung ist nicht in sämtlichen Phasen möglich, die Aufspürung jedoch in sämtlichen Phasen.

Im Rahmen unserer Vorgaben und mit der Orientierung am Killchain Modell, setzen wir zusätzlich die nachfolgenden Softwares ein, zur Verteidigung gegen potenzielle Cyberangriffe.

\begin{table}[ht]
    \begin{center}
        \begin{tabular}{|l|l|l|}
            \hline
            Phase        & Detektion      & Abwehr        \\ \hline
            Aufklärung   &                & \href{https://feodotracker.abuse.ch/blocklist/}{Firewall}            \\ \hline
            Angriff      &                & SSHGuard            \\ \hline
            Übermittlung &                & \href{https://gitlab.com/malware-filter/urlhaus-filter}{URLhaus Filter}     \\ \hline
            Ausnutzung   &                & Unattended Upgrades \\ \hline
            Installation & OSSEC          &                     \\ \hline
            C2           &                & Firewall            \\ \hline
            Manipulation & Auditd Service/OSSEC &                     \\ \hline
        \end{tabular}
    \end{center}
    \caption{Auswahl verschiedener Softwares zur Abwehr von Cyberangriffen}
    \label{tab:tools_killchain}
\end{table}

Die in \autoref{tab:tools_killchain} aufgelisteten Softwares wie z. B. SSHGuard oder Unattended Upgrades werden im \autoref{kap:haertung_des_systems} detailliert erläutert.