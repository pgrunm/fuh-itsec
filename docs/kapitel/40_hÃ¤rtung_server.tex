\newpage
\subsection{Härtung des Systems}\label{kap:haertung_des_systems}

% Literatur
% https://github.com/trimstray/the-practical-linux-hardening-guide#policy-compliance
% https://dev-sec.io/

\subsubsection{Empfehlungen aus dem BSI Grundschutz zur Härtung}

F\"ur die H\"artung des Servers m\"ussen die einzelnen verwendeten Anwendungen und Services betrachtet werden. Im vorliegenden Fall wird ein Webserver mit einer relationalen Datenbank betrieben, welcher mit kritischer Infrastruktur in Ber\"uhrung kommt. Bei IT-Systemen aus Sektoren, welche unter \S2 Abs. 10 Nr. 1 des BSI-Gesetz fallen (kritische Infrastruktur), muss zweij\"ahrig eine Selbsterkl\"arung zur IT-Sicherheit abgeben werden. Dies umfasst beispielsweise die Sektoren Energie, Informationstechnik und Telekommunikation, Transport und Verkehr, Gesundheit und weitere. In dieser Selbsterkl\"arung m\"ussen Zertifizierungen, Audits und konkrete Systeme zur Sicherung besonders sch\"utzenswerter Daten angegeben werden. F\"ur Zertifizierungen ist ein Informationssicherheitsmanagementsystem (ISMS) nach DIN EN ISO/IEC 27000 oder BSI IT-Grundschutz erforderlich. Durch das Bundesamt f\"ur Sicherheit in der Informationstechnik wird das IT-Grundschutzkompendium zur Verf\"ugung gestellt.  Einzelne Themen der IT-Sicherheit werden in sogenannten Bausteinen behandelt, welche im Kompendium zusammengefasst werden. In jedem Baustein wird Themenbezogen auf die Gef\"ahrdungen eingegangen und Anforderungen zu Absicherung empfohlen. Dadurch sollen die Grundwerte Vertraulichkeit, Verf\"ugbarkeit und Integrit\"at gew\"ahrleistet werden. Wichtig ist, dass die Ma{\ss}nahmen aus den Bausteinen des IT-Grundschutzes nicht als abschlie{\ss}end gesehen werden d\"urfen. Je nach Konfiguration und Gef\"ahrdungen m\"ussen weitere Ma{\ss}nahmen ber\"ucksichtigt werden. F\"ur die operative IT-Sicherheit k\"onnen diverse Onlinequellen wie beispielsweise https://dev-sec.io/ herangezogen werden. Die Ma{\ss}nahmen m\"ussen sich am Stand der Technik orientieren. \\

Folgende Bausteine aus dem IT-Grundschutz werden als relevant angesehen:
  \begin{itemize}
      \item APP.3.2 Webserver
			\item APP.4.3 Relationale Datenbanken
      \item SYS.1.3 Server unter Linux und Unix
			\item SYS.1.6 Containerisierung
   \end{itemize} 

Im Rahmen dieser Dokumentation werden die folgenden Anforderungen nur auszugsweise genannt. Die vollst\"andigen Bausteine mit den Anforderungen nach IT-Grundschutz werden vom Bundesamt f\"ur Sicherheit in der Informationstechnik kostenfrei zur Verf\"ugung gestellt.

\paragraph{Webserver}
\noindent \\Der Webserver nimmt Anfrage eines Clients entgegen und liefert die angefragten Webseiten aus. Folgende Anforderungen werden an der Webserver gestellt:
  \begin{itemize}
      \item APP.3.2.A1 Sichere Konfiguration eines Webservers\\
			Dazu muss man insbesondere den Webserver-Prozess einem Konto mit minimalen Rechten zuweisen. Der Webserver muss in einer gekapselten Umgebung ausgef\"uhrt werden, sofern dies vom Betriebssystem unterst\"utzt wird. Ist dies nicht m\"oglich, sollte jeder Webserver auf einem eigenen physischen oder virtuellen Server ausgef\"uhrt werden. Dem Webserver-Dienst m\"ussen alle nicht notwendige Schreibberechtigungen entzogen werden. Nicht ben\"otigte Module und Funktionen des Webservers m\"ussen deaktiviert werden.\cite{Grundschutz}\\
			
			\item APP.3.2.A2 Schutz der Webserver-Dateien\\
			Der IT-Betrieb muss alle Dateien auf dem Webserver, insbesondere Skripte und Konfigurationsdateien, so sch\"utzen, dass sie nicht unbefugt gelesen und ge\"andert werden k\"onnen. Es muss sichergestellt werden, dass Webanwendungen nur auf einen definierten Verzeichnisbaum zugreifen k\"onnen (WWW-Wurzelverzeichnis). Der Webserver muss so konfiguriert sein, dass er nur Dateien ausliefert, die sich innerhalb des WWW-Wurzelverzeichnisses befinden. Der IT-Betrieb muss alle nicht ben\"otigten Funktionen, die Verzeichnisse auflisten, deaktivieren. Vertrauliche Daten m\"ussen vor unberechtigtem Zugriff gesch\"utzt werden. Insbesondere muss der
IT-Betrieb sicherstellen, dass vertrauliche Dateien nicht in \"offentlichen Verzeichnissen des Webservers liegen. Der IT-Betrieb muss regelm\"a{\ss}ig \"uberpr\"ufen, ob vertrauliche Dateien in \"offentlichen Verzeichnissen gespeichert wurden.\cite{Grundschutz}\\

      \item APP.3.2.A4 Protokollierung von Ereignissen\\
			Der Webserver muss mindestens folgende Ereignisse protokollieren:
\begin{itemize}
\item erfolgreiche Zugriffe auf Ressourcen,
\item fehlgeschlagene Zugriffe auf Ressourcen aufgrund von mangelnder Berechtigung, nicht vorhandenen Ressourcen und Server-Fehlern sowie
\item allgemeine Fehlermeldungen.
\end{itemize}
Die Protokollierungsdaten sollten regelm\"a{\ss}ig ausgewertet werden.\cite{Grundschutz}\\
			
			\item APP.3.2.A11 Verschl\"usselung \"uber TLS\\
			Der Webserver muss f\"ur alle Verbindungen durch nicht vertrauensw\"urdige Netze eine sichere Verschl\"usselung \"uber \ac{TLS} anbieten (HTTPS). Falls es aus Kompatibilit\"atsgr\"unden erforderlich ist, veraltete Verfahren zu verwenden, sollten diese auf so wenige F\"alle wie m\"oglich beschr\"ankt werden. Wenn eine HTTPS-Verbindung genutzt wird, m\"ussen alle Inhalte \"uber HTTPS ausgeliefert werden. Sogenannter Mixed Content darf nicht verwendet werden.\cite{Grundschutz}\\
			
			\item APP.3.2.A14 Integrit\"atspr\"ufungen und Schutz vor Schadsoftware\\
			Der IT-Betrieb sollte regelm\"a{\ss}ig pr\"ufen, ob die Konfigurationen des Webservers und die von ihm bereitgestellten Dateien noch integer sind und nicht durch Angriffe ver\"andert wurden. Die zur Ver\"offentlichung vorgesehenen Dateien sollten regelm\"a{\ss}ig auf Schadsoftware gepr\"uft werden.\cite{Grundschutz}\\
						
   \end{itemize}  

\paragraph{Relationale Datenbanken}
\noindent \\Eine Datenbank verwaltet gro{\ss}e Datenmengen und ist im Zusammenhang mit einem Webserver f\"ur die dauerhafte Speicherung von Informationen und Inhalten der Webseite erforderlich. Folgende Anforderungen werden an die Datenbank gestellt:

\begin{itemize}
			\item  APP.4.3.A13 Restriktive Handhabung von Datenbank-Links \\
			Es sollte sichergestellt sein, dass nur Zust\"andige dazu berechtigt sind, Datenbank-Links (DB-Links) anzulegen. Werden solche Links angelegt, m\"ussen so genannte Private DB-Links vor Public DB-Links bevorzugt angelegt werden. Alle von den Zust\"andigen angelegten DB-Links SOLLTEN dokumentiert und regelm\"a{\ss}ig \"uberpr\"uft werden. Zudem sollten DB-Links mitber\"ucksichtigt werden, wenn das Datenbanksystem gesichert wird (siehe APP.4.3.A9 Datensicherung eines Datenbanksystems).\cite{Grundschutz}\\

			\item APP.4.3.A16 Verschl\"usselung der Datenbankanbindung \\
			Das Datenbankmanagementsystem sollte so konfiguriert werden, dass Datenbankverbindungen immer verschl\"usselt werden. Die dazu eingesetzten kryptografischen Verfahren und Protokolle sollten den internen Vorgaben der Institution entsprechen (siehe CON.1 Krypto Konzept).\cite{Grundschutz}\\
			
			\item APP.4.3.A21 Einsatz von Datenbank Security Tools\\
			Es sollten Informationssicherheitsprodukte f\"ur Datenbanken eingesetzt werden. Die eingesetzten Produkte sollten folgende Funktionen bereitstellen:
\begin{itemize}
	\item Erstellung einer \"Ubersicht \"uber alle Datenbanksysteme,
	\item erweiterte Konfigurationsm\"oglichkeiten und Rechtemanagement der Datenbanken,
	\item Erkennung und Unterbindung von m\"oglichen Angriffen (z. B. Brute Force Angriffe auf Konten, SQL-Injection) und
	\item Auditfunktionen (z. B. \"Uberpr\"ufung von Konfigurationsvorgaben).\cite{Grundschutz}\\
\end{itemize}
			
			\item APP.4.3.A24 Datenverschl\"usselung in der Datenbank\\
			Die Daten in den Datenbanken sollten verschl\"usselt werden. Dabei sollten vorher unter anderem folgende Faktoren betrachtet werden:
\begin{itemize}
	\item Einfluss auf die Performance,
	\item Schl\"usselverwaltungsprozesse und -verfahren, einschlie{\ss}lich separater Schl\"usselaufbewahrung und -sicherung,
	\item Einfluss auf Backup-Recovery-Konzepte,
	\item funktionale Auswirkungen auf die Datenbank, beispielsweise Sortierm\"oglichkeiten.\cite{Grundschutz}\\
\end{itemize}
\end{itemize}


\paragraph{Server unter Linux}
\noindent \\Linux stellt im vorliegenden Fall das Serverbetriebssytem dar. Bei Linux handelt es sich um eine freie Open-Source-Software. Folgende Anforderungen werden an das Betriebssystem Linux gestellt:

\begin{itemize}
	\item	SYS.1.3.A2 Sorgf\"altige Vergabe von IDs\\ 
	Jeder Login-Name, jede User-ID (UID) und jede Gruppen-ID (GID) darf nur einmal vorkommen. Jedes Konto von Benutzenden muss Mitglied mindestens einer Gruppe sein. Jede in der Datei /etc/passwd vorkommende GID muss in der Datei /etc/group definiert sein. Jede Gruppe sollte nur die Konten enthalten, die unbedingt notwendig sind. Bei vernetzten Systemen muss au{\ss}erdem darauf geachtet werden, dass die Vergabe von Benutzenden- und Gruppennamen, UID und GID im Systemverbund konsistent erfolgt, wenn beim system\"ubergreifenden Zugriff die M\"oglichkeit besteht, dass gleiche UIDs bzw. GIDs auf den Systemen unterschiedlichen Benutzenden- bzw. Gruppennamen zugeordnet werden k\"onnten.\cite{Grundschutz}\\
	
	\item SYS.1.3.A4 Schutz vor Ausnutzung von Schwachstellen in Anwendungen\\
	Um die Ausnutzung von Schwachstellen in Anwendungen zu erschweren, muss ASLR und DEP/NX im Kernel aktiviert und von den Anwendungen genutzt werden. Sicherheitsfunktionen des Kernels und der Standardbibliotheken, wie z. B. Heap- und Stackschutz, d\"urfen nicht deaktiviert werden.\cite{Grundschutz}\\
	
	\item SYS.1.3.A8 Verschl\"usselter Zugriff \"uber Secure Shell\\
	Um eine verschl\"usselte und authentisierte, interaktive Verbindung zwischen zwei IT-Systemen aufzubauen, sollte ausschlie{\ss}lich Secure Shell (SSH) verwendet werden. Alle anderen Protokolle, deren Funktionalit\"at durch Secure Shell abgedeckt wird, sollten vollst\"andig abgeschaltet werden. F\"ur die Authentifizierung sollten vorrangig Zertifikate anstatt eines Passworts verwendet werden.\cite{Grundschutz}\\

	\item SYS.1.3.A10 Verhinderung der Ausbreitung bei der Ausnutzung von Schwachstellen\\
	Dienste und Anwendungen sollten mit einer individuellen Sicherheitsarchitektur gesch\"utzt werden (z. B. mit AppArmor oder SELinux). Auch chroot-Umgebungen sowie LXC- oder Docker-Container sollten dabei ber\"ucksichtigt werden. Es sollte sichergestellt sein, dass mitgelieferte Standardprofile bzw. -regeln aktiviert sind.\cite{Grundschutz}\\
\end{itemize}
 

\paragraph{Containerisierung}
\noindent \\Der Begriff Containerisierung bezeichnet ein Konzept, bei dem Ressourcen eines Betriebssystems partitioniert werden, um Ausf\"uhrungsumgebungen f\"ur Prozesse zu schaffen. Hierbei k\"onnen je nach verwendetem Betriebssystem unterschiedliche Techniken zum Einsatz kommen, die sich in Funktionsumfang und Sicherheitseigenschaften unterscheiden. Oft wird auch von einer "`Betriebssystemvirtualisierung"' gesprochen. Es wird jedoch kein vollst\"andiges Betriebssystem
virtualisiert, sondern es werden lediglich bestimmte Ressourcen durch einen geteilten Kernel zur Verf\"ugung gestellt. Folgende Anforderungen sind bei einer Containerisierung relevant:

\begin{itemize}
	\item SYS.1.6.A7 Persistenz von Protokollierungsdaten der Container\\
	Die Speicherung der Protokollierungsdaten der Container muss au{\ss}erhalb des Containers, mindestens auf dem Container-Host, erfolgen.\cite{Grundschutz}\\
	
	\item SYS.1.6.A8 Sichere Speicherung von Zugangsdaten bei Containern \\
	Zugangsdaten m\"ussen so gespeichert und verwaltet werden, dass nur berechtigte Personen und Container darauf zugreifen k\"onnen. Insbesondere muss sichergestellt sein, dass Zugangsdaten nur an besonders gesch\"utzten Orten und nicht in den Images liegen. Die von der Verwaltungssoftware des Container-Dienstes bereitgestellten Verwaltungsmechanismen f\"ur Zugangsdaten sollten eingesetzt werden. Mindestens die folgenden Zugangsdaten m\"ussen sicher gespeichert werden:
\begin{itemize}
	\item Passw\"orter jeglicher Accounts,
	\item API-Keys f\"ur von der Anwendung genutzte Dienste,
	\item Schl\"ussel f\"ur symmetrische Verschl\"usselungen sowie
	\item private Schl\"ussel bei Public-Key-Authentisierung.\cite{Grundschutz}\\
\end{itemize}	

	\item SYS.1.6.A24 Hostbasierte Angriffserkennung \\
	Das Verhalten der Container und der darin betriebenen Anwendungen und Dienste sollte \"uberwacht werden. Abweichungen von einem normalen Verhalten sollten bemerkt und gemeldet werden. Die Meldungen sollten im zentralen Prozess zur Behandlung von Sicherheitsvorf\"allen angemessen behandelt werden. Das zu \"uberwachende Verhalten solte mindestens umfassen:
\begin{itemize}
	\item Netzverbindungen,
	\item erstellte Prozesse,
	\item Dateisystem-Zugriffe und
	\item Kernel-Anfragen (Syscalls).\cite{Grundschutz}\\
\end{itemize}
	
\end{itemize}

   
\subsubsection{Härtung der installierten Software}

% TODO: Guten Content niederschreiben.
Die Härtung des Systems und der installierten Dienste erfolgt durch das im \autoref{listing:hardening} dargestellten Ansible Playbook. Bei der Härtung wird unter anderem auf eine Ansible Rolle des \enquote{Dev-Sec} Projekts gesetzt, welches eine grundsätzliche Härtung auf der Basis von Empfehlungen der Deutschen Telekom, der Webseite \href{https://bettercrypto.org/}{Bettercrypto} sowie der amerikanischen \ac{NSA}\footcite[Vgl.][]{OverviewDevSecBaselines}. 

% Mehr vergleichen
Eine Eigenentwicklung auf Basis der Vorgaben des \ac{BSI} oder anderer Empfehlungen wäre zwar auch möglich, jedoch weiter weniger umfassend und ausgiebig getestet. Es stünden auch alternative Projekte bereit, wie beispielsweise das \enquote{ComplianceAsCode} Projekt, dessen Installation bzw. Konfiguration mehr Aufwand beansprucht als eine Ansible Rolle.

% Ansible Rolle erläutern
Das Dev-Sec Projekt bietet eine vorgefertigte Ansible Rolle an, welches exakt zu den Vorgaben der FernUni Inc. passt. Sämtliche ausgewählten Softwarekomponenten werden durch die Rolle unterstützt und können somit in eine umfassende und sichere Grundkonfiguration versetzt werden. 
Eine ausführliche Auflistung sämtlicher Einstellungen ist auf der Seite des Dev-Sec Projekts ersichtlich\footcite[Vgl.][]{OverviewDevSecBaselines}.

Die Ansible Rolle legt sichere Standard Konfigurationswerte unter anderem für das Betriebssystem, \ac{SSH}, \ac{TLS}, MySQL und Nginx fest, die wiederum durch eigene zusätzliche Einstellungen noch ergänzt werden. In Summe ergeben diese ein funktionierendes und sicheres Gesamtsystem. Durch die Kombination mit Ansible können die definierten Einstellungen schnell und unkompliziert auf einer Vielzahl von Server angewendet werden.

% Betriebssystem
Innerhalb des Betriebssystems werden durch die Ansible Rollen verschiede Einstellungen vorgenommen, darunter die folgenden besonders relevanten:

\begin{itemize}
	\item Änderung der Besitzerrechte am Verzeichnis \enquote{/var/log} auf Root, sodass nur dieser Benutzer dort Veränderungen vornehmen kann.
	\item Installation des \enquote{Syslog} Dienstes, um eine Protokollierung zu ermöglichen.
	\item Installation des \enquote{Auditd} Dienstes, um erweiterte Protokollierung zu ermöglichen.
	\item Deaktivierung der Weiterleitung von \ac{IP} Paketen.
	\item Überprüfung der Zugangsrechte für die Dateien \enquote{/etc/shadow} und \enquote{/etc/passwd}.
\end{itemize}

Die gelisteten Maßnahmen stellen lediglich einen Teil der Härtung innerhalb des Betriebssystems dar. Sie verdeutlichen jedoch die Notwendigkeit sichere Grundeinstellungen festzulegen. Insbesondere der Schutz der Protokollierungsdaten und deren Erstellung sind von einer hohen Bedeutung. Sollte einem Angreifer der Zugriff auf das System gelingen, wäre dies darüber nachvollziehbar.

% Weitere Erläuterungen zum Thema OS Härtung
Weiterhin wird unter anderem verhindert, dass beispielsweise \ac{IP} Pakete weitergeleitet werden, um beispielsweise die Aufklärung des Netzwerks zu verhindern.

% Auf Härtung SSH eingehen
Zusätzlich zur Härtung des Betriebssystems ist auch der Zugang zum Server von Interesse. Dieser erfolgt ausschließlich aus der Ferne via \ac{SSH}. Dieser Dienst ist selbstverständlich auch zu Härten, eine Auswahl der Maßnahmen bietet die folgende Übersicht:

\begin{itemize}
	\item Konfiguration der verwendeten SSH Version 2, weil Version 1 Sicherheitslücken aufweist.
	\item Deaktivierung des Logins via Passwort, stattdessen ist lediglich ein Login via SSH Schlüssel erlaubt.
	\item Die Deaktivierung von SSH Tunnels, sodass kein Port Forwarding möglich ist.
	\item Deaktivierung des Root Logins, sodass sich lediglich nicht Root Benutzer anmelden dürfen.
\end{itemize}

% SSH erläutern
Die Härtung der SSH Konfiguration zielt darauf auf, beispielsweise sich per Root Benutzer anmelden zu können. Somit werden weitgehende Administratorrechte direkt ab der Anmeldung unterbunden. Zur Authentisierung werden lediglich \ac{SSH} Schlüssel zugelassen, sodass keine Anmeldung per Kennwort mehr möglich ist. Dies erschwert zusätzlich den Remote Zugang zum System und steigert insbesondere mit den vorgegebenen Firewall Regeln sowie SSHGuard die Hürde zum Zugriff beträchtlich.

% MySQL erläutern
Zu guter Letzt steht noch die Härtung des Datenbankmanagementsystems an. Da dieses sämtliche Daten speichert, sind diese auch von großem Wert für einen Angreifer. Eine Auswahl der Maßnahmen zur Härtung stellen die folgenden dar:

\begin{itemize}
	\item MySQL Datenverzeichnis bzw. Log Verzeichnis gehört dem MySQL User.
	\item Änderung der Standardpasswörter.
	\item MySQL Konfigurationsdatei besitzt der Root Benutzer.
\end{itemize}

Die notwendigen Einstellungen beim Zugang zu MySQL sind recht übersichtlich, sodass sich diese primär auf Manipulation beim Zugang zu Konfigurationsdateien beziehen.

%% Weitere Tools: SSHGuard %%
% Einige Einstellungen erläutern
Dies umfasst neben den bereits erwähnten Grundeinstellungen beispielsweise die Einrichtung von SSHGuard, welches nach drei fehlerhaften Anmeldeversuchen die Quell IP Adresse für drei Minuten blockiert:

\begin{minted}[]{properties}
	# Block an attacker after 3 failed logins.
	THRESHOLD=30
	# Blocks for at least 3mins.
	BLOCK_TIME=180
	# Attackers are remembered for 10m.
	DETECTION_TIME=3600
\end{minted}
% \caption{Ausschnitt der SSHGuard Konfigurationsdatei}

In der Standardeinstellung werden weitere Anfragen mit dem Faktor 1.5 gesperrt, sodass die Blockierung kontinuierlich steigt, wie in \autoref{listing:sshguard_log} dargestellt. Dies bremst Angreifer aus, die versuchen den Server per SSH anzugreifen. Zusätzlich muss der Angriff aus dem internen Netz erfolgen, da die Firewall nur Zugriff von intern erlaubt. Mittels einer Einstellung im Ansible Playbook lässt sich diese Einstellung variabel anpassen, um beispielsweise auf Angriffe zu reagieren.

In Verbindung mit der Firewall und SSHGuard stellen beide Maßnahmen bereits eine entscheidende Hürde für potenzielle Angreifer dar, deren Überwindung Zeit und Aufwand erfordern.

Die einzigen anderen Möglichkeiten, Zugriff von außen zu erlangen stellen der Webserver Nginx sowie das \ac{VPN} Wireguard dar.

% TODO: Nginx auf non Root umstellen
% s. hier: https://www.sherbers.de/running-nginx-without-root/
Standardmäßig läuft der Nginx Master Prozess unter dem Benutzer \enquote{Root} und sollte durch eine Sicherheitslücke der Webserver anfällig sein, wäre der Webserver als kompromittiert zu betrachten. Dementsprechend ist Nginx umgestellt, dass er nicht mehr als Root Benutzer läuft.

Zusätzlich unterliegen sowohl Nginx als auch Wireguard einem AppArmor Profil, wie in \autoref{kap:installation_apparmor} beschrieben. Dies beschränkt den Zugang zum System weiter.

% Auf OSSEC eingehen.
Zusätzlich zu den beschriebenen Härtungsmaßnahmen sind weitere Maßnahmen von \enquote{Innen} implementiert. Dazu zählen beispielsweise die Nutzung automatischer Aktualisierungen durch \enquote{Unattended Upgrades}, welches automatisch Sicherheitsupdates installiert.

% https://www.redhat.com/sysadmin/configure-linux-auditing-auditd
Sollte es dennoch einem Angreifer gelingen, Zugriff auf den Server gelangen, sind dort noch zusätzlich der Auditd Service und OSSEC als \ac{IDS} installiert. Auditd protokolliert sämtliche Aktionen in der Log Datei \mint[]{shell}|/var/log/audit/audit.log| sodass sämtliche Aktionen nachvollzogen werden können.

Zusätzlich bietet OSSEC an neben neben der Intrusion Detection auch Dateien auf Änderungen zu Überwachen. Dies ermöglicht die Alarmierung eines Administrators, sofern Manipulationen festgestellt werden.

% TODO: Integration OSSEC mit Prometheus Node Exporter? Alternative: Eigener Exporter :think:
% Dokumentation: https://www.ossec.net/docs/docs/manual/output/json-alert-log-output.html

Beides basiert auf den Überlegungen des Cyber-Killchain Modells und stellt eine hybride Strategie aus Detektion und Abwehr dar, um Angriffe entweder abzuwehren oder aufspüren zu können. Je mehr Aufwand ein Angreifer aufwenden muss, desto wahrscheinlicher ist auch die Detektion durch die getroffenen Maßnahmen.