\newpage
\subsubsection{H\"artung des Servers}

F\"ur die h\"artung des Servers m\"ussen die einzelnen verwendeten Anwendung und Services betrachtet werden. Im vorliegenden Fall wird ein Webserver mit einer relationalen Datenbank betrieben. Daneben werden Services wie SSH und ein VPN-Server zur Verf\"ugung gestellt. Neben diversen Online-Quellen wie beispielsweise https://dev-sec.io/ wird durch das Bundesamt f\"ur Sicherheit in der Informationstechnik das IT-Grundschutzkompendium zur Verf\"ugung gestellt. Einzelne Themen der IT-Sicherheit werden in sogenannten Bausteinen behandelt, welche im Kompendium zusammengefasst werden. In jedem Baustein wird Themenbezogen auf die Gef\"ahrdungen eingegangen und Ma{\ss}nahmen zu Absicherung empfohlen. Dadurch sollen die Grundwerte Vertraulichkeit, Verf\"ugbarkeit und Integrit\"at gew\"ahrleistet werden. Wichtig ist, dass die Ma{\ss}nahmen aus den Bausteinen des IT-Grundschutzes nicht als abschlie{\ss}end gesehen werden d\"urfen. Je nach Konfiguration und Gef\"ahrdungen m\"ussen weitere Ma{\ss}nahmen ber\"ucksichtigt werden. \\

Folgende Bausteine aus dem IT-Grundschutz werden als relevant angesehen:
  \begin{itemize}
      \item APP.3.2 Webserver
			\item APP.4.3 Relationale Datenbanken
      \item SYS.1.3 Server unter Linux und Unix
			\item SYS.1.6 Containerisierung
   \end{itemize} 

\paragraph{Webserver}
\noindent \\Der Webserver  

\paragraph{Relationale Datenbanken}
\noindent \\Der Webserver  

\paragraph{Server unter Linux}
\noindent \\Der Webserver  

\paragraph{Containerisierung}
\noindent \\Der Webserver  