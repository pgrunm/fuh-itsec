\section{Organisatorischer Überblick}

\subsection{Gruppenorganisation}

% \begin{itemize}
%     \item Welcher Tools nutzen wir?
%     \item Wie wurden die Aufgaben verteilt und wie wurden diese gelöst?
%     \item Welche Hindernisse sind bei uns aufgetreten und wie haben wir diese gelöst?
% \end{itemize}

Die Organisation unserer Gruppe erfolgte insbesondere online. Besprechungen erfolgten via Discord, kurze Abstimmungen beziehungsweise Statusmeldungen in einer eigens eingerichteten Whatsapp Gruppe.

% Kanban Board
Für die Organisation und Verteilung sämtlicher Aufgaben kam ein eigens dafür eingerichtetes Kanban Board der quelloffenen Kollaborations Plattform \enquote{CryptPad} zum Einsatz. Mithilfe des Boards konnte jeder nachvollziehen welche Aufgaben bereits bearbeitet werden oder noch zu erledigen sind und sich selbstständig eine Aufgabe aussuchen.

% Quellcode erläutern
Um ein dezentrales Arbeiten zu ermgöglichen liegen sowohl der Quellcode der Dokumentation als auch sämtliche Konfigurationsdateien in einem Git Repository auf der Quellcode Hosting Plattform Github. Dies ermöglicht jedem Gruppenmitglied jederzeit Zugang und erlaubt zudem eine Versionierung.