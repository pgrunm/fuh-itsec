\section{Anhang}

\subsection{Installationsskript AppArmor}

\begin{listing}[ht]{}
    \shellcode{code/shell/apparmor_profile.sh}
    \caption{Installation der AppArmor Profile}
    \label{listing:installation_apparmor}
\end{listing}

\newpage
\subsection{Ansible Playbooks}

\subsubsection{Installation benötigter Ansible Module}
\begin{listing}
    \yamlcode{code/ansible/requirements.yaml}
    \caption{Installation erforderlicher Ansible Module}
    \label{listing:ansible_module_installation}
\end{listing}


\subsubsection{Ansible Playbook zur Einrichtung des Systems}
\begin{longlisting}
    \yamlcode{code/ansible/playbook.yaml}
    \caption{Installation des Servers mit Ansible}
    \label{listing:ansible_playbook}
\end{longlisting}

\subsubsection{Ansible Playbook zur Härtung des Systems}
\begin{longlisting}
    \yamlcode{code/ansible/hardening.yaml}
    \caption{Härtung des Servers mit Ansible}
    \label{listing:hardening}
\end{longlisting}

\subsection{Ausgabe von Konfigurations- und Logdateien}

\subsubsection{Ausgabe eines unterbundenen SSH Brute Force Angriffs}
\inputminted[breaklines=true,firstline=34, lastline=42]{text}{code/output/sshguard.txt}\label{listing:sshguard_log}
