\section{Anhang}

\subsection{Containerisierung der Python Programme}

\subsubsection{Hello Programm}

\inputminted[fontfamily=tt,
linenos=true,
numberblanklines=true,
numbersep=5pt,
gobble=0,
frame=leftline,
framerule=0.4pt,
framesep=2mm,
funcnamehighlighting=true,
tabsize=4,
obeytabs=false,
mathescape=false
samepage=false, %with this setting you can force the list to appear on the same page
showspaces=false,
showtabs =false,
texcl=false,
breaklines=true]{docker}{code/docker/hello/Dockerfile}

\subsubsection{Messages Programm}
\inputminted[fontfamily=tt,
linenos=true,
numberblanklines=true,
numbersep=5pt,
gobble=0,
frame=leftline,
framerule=0.4pt,
framesep=2mm,
funcnamehighlighting=true,
tabsize=4,
obeytabs=false,
mathescape=false
samepage=false, %with this setting you can force the list to appear on the same page
showspaces=false,
showtabs =false,
texcl=false,
breaklines=true]{docker}{code/docker/messages/Dockerfile}

\subsection{Konfigurationsdateien Nginx}\label{kap:nginx}

\subsubsection{Konfiguration \enquote{hello.fernuni}}
\nginxcode[]{code/nginx/sites-available/hello.fernuni.conf}

\subsubsection{Konfiguration \enquote{messages.fernuni}}
\nginxcode[]{code/nginx/sites-available/messages.fernuni.conf}

\subsubsection{Konfiguration Nginx}
\nginxcode[]{code/nginx/nginx.conf}

\subsection{AppArmor}
\subsubsection{Installationsskript AppArmor}


\begin{longlisting}[ht]{}
    \shellcode{code/shell/apparmor_profile.sh}
    \caption{Installation der AppArmor Profile}
    \label{listing:installation_apparmor}
\end{longlisting}

\subsubsection{Ausgabe des AppArmor Logs}

\begin{longlisting}
    \inputminted[fontfamily=tt,
    linenos=true,
    numberblanklines=true,
    numbersep=5pt,
    gobble=0,
    frame=leftline,
    framerule=0.4pt,
    framesep=2mm,
    funcnamehighlighting=true,
    tabsize=4,
    obeytabs=false,
    mathescape=false
    samepage=false, %with this setting you can force the list to appear on the same page
    showspaces=false,
    showtabs =false,
    texcl=false,
    breaklines=true,
    lastline=10]{text}{code/shell/apparmor.log}
    \caption{Ausgabe des Apparmor Logs}
    \label{listing:log_apparmor}
\end{longlisting}
\newpage
\subsection{Ansible Playbooks}

\subsubsection{Installation benötigter Ansible Module}
\begin{listing}[ht]
    \yamlcode{code/ansible/requirements.yaml}
    \caption{Installation erforderlicher Ansible Module}
    \label{listing:ansible_module_installation}
\end{listing}


\subsubsection{Ansible Playbook zur Einrichtung des Systems}
\begin{longlisting}
    \yamlcode{code/ansible/playbook.yaml}
    \caption{Installation des Servers mit Ansible}
    \label{listing:ansible_playbook}
\end{longlisting}

\subsubsection{Ansible Playbook zur Härtung des Systems}
\begin{longlisting}
    \yamlcode{code/ansible/hardening.yaml}
    \caption{Härtung des Servers mit Ansible}
    \label{listing:hardening}
\end{longlisting}

\subsection{Ausgabe von Konfigurations- und Logdateien}

\subsubsection{Blockierung bösartiger IP Adressen}\label{kap:fw_abuse}

\inputminted[fontfamily=tt,
linenos=true,
numberblanklines=true,
numbersep=5pt,
gobble=0,
frame=leftline,
framerule=0.4pt,
framesep=2mm,
funcnamehighlighting=true,
tabsize=4,
obeytabs=false,
mathescape=false
samepage=false, %with this setting you can force the list to appear on the same page
showspaces=false,
showtabs =false,
texcl=false,
breaklines=true]{docker}{code/shell/fw_abuse.sh}

\subsubsection{Ausgabe eines unterbundenen SSH Brute Force Angriffs}
\inputminted[breaklines=true,firstline=34, lastline=42]{text}{code/output/sshguard.txt}\label{listing:sshguard_log}

\subsection{Python Testskript}\label{kap:testskript_python}

\pythoncode{code/python/test_server.py}