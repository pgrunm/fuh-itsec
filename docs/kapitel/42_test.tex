\subsection{Test des Systems}

Um die Funktionalität des Gesamtsystems zu gewährleisten, wurde die Konfiguration ausgiebig getestet. Dies geschah einerseits manuell durch die Autoren bei der Installation sowie Konfiguration des Systems.

Zusätzlich zu manuell durchgeführten Tests haben die Autoren einige automatische Tests in Python konzipiert, die mittels des Moduls \enquote{Testinfa} einige Vorgaben verifizieren. Eine Ausgabe des Testskripts sieht wie folgt aus:

\begin{minted}[breaklines=true]{text}
    ============================= test session starts ==============================
platform linux -- Python 3.11.3, pytest-7.4.0, pluggy-1.2.0
rootdir: ~/fuh-itsec
plugins: testinfra-8.1.0
collected 11 items

docs/code/python/test_server.py .......                                  [100%]

============================== 11 passed in 1.30s ===============================
\end{minted}

Durch die in \autoref{kap:testskript_python} dargestellten Tests wird der Server automatisiert getestet. Dabei wird geprüft, ob:

\begin{itemize}
    \item Nginx installiert bzw. aktiviert ist und läuft.
    \item Ein MariaDB Server installiert bzw. aktiviert ist und läuft.
    \item Die Pakete SSHGuard, AppArmor, OSSEC, Docker und libmariadb-dev installiert sind.
    \item Einige Dateien und Verzeichnisse existieren.
    \item Verbindung auf den TCP Ports 80, 443 und 3306 möglich sind.
\end{itemize}

Die Tests sind dabei so konzipiert, dass direkt erkenntbar ist, falls einer fehlschlägt.